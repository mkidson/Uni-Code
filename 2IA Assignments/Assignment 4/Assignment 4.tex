\documentclass[12pt]{article}
\usepackage[margin=1.2in]{geometry}
\usepackage[all]{nowidow}
\usepackage[hyperfigures=true, hidelinks, pdfhighlight=/N]{hyperref}
\usepackage[separate-uncertainty=true,group-digits=false]{siunitx}
\usepackage{graphicx,amsmath,physics,tabto,float,amssymb,pgfplots,verbatim,tcolorbox}
\usepackage{listings,xcolor,subfig,keyval2e,caption,import}
\numberwithin{equation}{section}
\numberwithin{figure}{section}
\definecolor{stringcolor}{HTML}{C792EA}
\definecolor{codeblue}{HTML}{2162DB}
\definecolor{commentcolor}{HTML}{4A6E46}
\lstdefinestyle{appendix}{
    basicstyle=\ttfamily\footnotesize,commentstyle=\color{commentcolor},keywordstyle=\color{codeblue},
    stringstyle=\color{stringcolor},showstringspaces=false,numbers=left,upquote=true,captionpos=t,
    abovecaptionskip=12pt,belowcaptionskip=12pt,language=Python,breaklines=true,frame=single}
\lstdefinestyle{inline}{
    basicstyle=\ttfamily\footnotesize,commentstyle=\color{commentcolor},keywordstyle=\color{codeblue},
    stringstyle=\color{stringcolor},showstringspaces=false,numbers=left,upquote=true,frame=tb,
    captionpos=b,language=Python}
\renewcommand{\lstlistingname}{Appendix}
\pgfplotsset{compat=1.17}

\title{Assignment 4}
\author{KDSMIL001 \; MAM2000W 2IA}
\date{\textbf{12 October 2020}}

\begin{document}
    \maketitle
    \begin{enumerate}
        \setcounter{enumi}{6}
        \item \begin{enumerate}
            \item True. Given some $\sigma=(k_1\, k_2)(k_3\, k_4)\dots(k_{r-1}\, k_r)$ which is a product of disjoint transpositions, 
            we know that the inverse of a transposition is the transposition itself, so we can write
            \begin{align*}
                \sigma^{-1}&=((k_1\, k_2)(k_3\, k_4)\dots(k_{r-1}\, k_r))^{-1}\\
                &=(k_1\, k_2)^{-1}(k_3\, k_4)^{-1}\dots(k_{r-1}\, k_r)^{-1}\\
                &=(k_1\, k_2)(k_3\, k_4)\dots(k_{r-1}\, k_r)\\
                &=\sigma
            \end{align*}
            \newline
            \begin{flushright}$\square$\end{flushright}
            \item True. Any permutation in $S_n$, aside from the identity permutation $\epsilon$, is not disjoint with its inverse. Firstly $\epsilon^{-1}=\epsilon$ 
            and $\epsilon$ fixes every value so its inverse fixes every element, making them disjoint as neither move the same element. Then any other element in $S_n$ 
            will be a product of one or more disjoint cycles of length at least 2 (by the Cycle Decomposition Theorem) and any cycle is not disjoint with its inverse since 
            the inverse \textit{must} move the same elements as the original cycle in order to reverse the moves.
            \newline
            \begin{flushright}$\square$\end{flushright}
        \end{enumerate}
    \end{enumerate}

\end{document}
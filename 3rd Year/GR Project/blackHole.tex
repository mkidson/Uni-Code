\documentclass[11pt]{article}
\usepackage[margin=1in, top=1in]{geometry}
\usepackage[all]{nowidow}
\usepackage[hyperfigures=true, hidelinks, pdfhighlight=/N]{hyperref}
\usepackage[separate-uncertainty=true, group-digits=false]{siunitx}
\usepackage{graphicx,amsmath,physics,tabto,float,amssymb,pgfplots,verbatim,tcolorbox}
\usepackage{listings,xcolor,subfig,caption,import,wrapfig,bbm}
\usepackage[version=4]{mhchem}
\usepackage[noabbrev]{cleveref}
\newcommand{\creflastconjunction}{, and\nobreakspace}
\numberwithin{equation}{section}
\numberwithin{figure}{section}
\numberwithin{table}{section}
\definecolor{stringcolor}{HTML}{C792EA}
\definecolor{codeblue}{HTML}{2162DB}
\definecolor{commentcolor}{HTML}{4A6E46}
\captionsetup{font=small, belowskip=0pt}
\lstdefinestyle{appendix}{
    basicstyle=\ttfamily\footnotesize,commentstyle=\color{commentcolor},keywordstyle=\color{codeblue},
    stringstyle=\color{stringcolor},showstringspaces=false,numbers=left,upquote=true,captionpos=t,
    abovecaptionskip=12pt,belowcaptionskip=12pt,language=Python,breaklines=true,frame=single}
\lstdefinestyle{inline}{
    basicstyle=\ttfamily\footnotesize,commentstyle=\color{commentcolor},keywordstyle=\color{codeblue},
    stringstyle=\color{stringcolor},showstringspaces=false,numbers=left,upquote=true,frame=tb,
    captionpos=b,language=Python}
\renewcommand{\lstlistingname}{Appendix}
\pgfplotsset{compat=1.17}

\begin{document}

\title{Extracting energy from black holes (theoretically)}
\author{Miles Kidson \\ Department of Mathematics and Applied Mathematics \\ University of Cape Town \\ Rondebosch 7701 \\ Cape Town, South Africa \\ Supervised by Prof. Peter Dunsby }
\date{October 2021}

\maketitle

\begin{figure}[h]
    \begin{center}
        \includegraphics{Plots/UCT.jpg}
    \end{center}
\end{figure}


\begin{center}
    \begin{abstract}
        In this paper we investigate a process similar to the Penrose process, which shows that a scalar wave travelling across the boundary of the ergosphere of a rotating, uncharged black hole will gain energy as it moves away from the black hole.
    \end{abstract}
\end{center}

\newpage
\section{Introduction}\label{sec:Introduction}
\par The underlying phenomenon that this paper investigates, that of black holes losing energy, is a phenomenon that at first glance doesn't seem to make much sense. We are told that black holes are the great attractor, letting nothing escape their grasp, not even light. Yet it turns out that the mathematics that led us to the concept of black holes, Einstein's general theory of relativity, also leads us towards the idea that, under the right circumstances, black holes can actually lose their energy. 
\par This process has been studied many times, most famously under the name the Penrose Process \cite{Penrose}. This paper will deviate from that process slightly by considering the use of a scalar wave to carry the energy from the black hole, as opposed to a massive object. We will, however, consider the same class of black hole: a rotating, uncharged black hole. The best description of this type of black hole comes from the Kerr metric \cite{kerr}, which is simply the Kerr-Newmann metric with the charge set to zero. 
\par In order to study this phenomenon, first we must build up the framework of general relativity that describes the specific circumstances where this effect can be seen. 

\section{Building the coordinate system}\label{sec:coordinates}
\par Since we are dealing with a system that is spherically symmetric, it's convenient for us to use a coordinate system that simplifies things. The simplest choice is spherical polar coordinates, where the transformation from cartesian coordinates is described by 
\begin{align}
    x=r\sin\theta\cos\phi\hspace{20pt}y&=r\sin\theta\sin\phi\hspace{20pt}z=r\cos\theta
    \label{eqn:cartesian to spherical polars}\\
    r=\sqrt{x^2+y^2+z^2}\hspace{20pt}\theta&=\arctan(\frac{\sqrt{x^2+y^2}}{z})\hspace{20pt}\phi=\arctan(y/x)
    \label{eqn:spherical polars to cartesian}
\end{align}
We can then find the transformation matrix $\Lambda_\beta^{\bar\alpha}$ such that $dx^{\bar\alpha}=\Lambda^{\bar\alpha}_\beta dx^\beta$. This involves expanding the differentials $dr$, $d\theta$, and $d\phi$ by chain rule and then computing the derivatives that show up. The $\Lambda_\beta^{\bar\alpha}$ matrix is then found to be
\begin{equation}
    \Lambda_\beta^{\bar\alpha}=
    \begin{pmatrix}
        \sin\theta\cos\varphi&\sin\theta\sin\varphi&\cos\theta\\
        \frac{\cos\theta\cos\varphi}{r}&\frac{\cos\theta\sin\varphi}{r}&-\frac{\sin\theta}{r}\\
        -\frac{\sin\varphi}{r\sin\theta}&\frac{\cos\varphi}{r\sin\theta}&0
    \end{pmatrix}
\end{equation}
We can define the one-form for an arbitrary scalar field $\varphi$:
\begin{equation}
    \tilde{\textbf{d}}\varphi\rightarrow\left(\frac{\partial\varphi}{\partial r},\frac{\partial\varphi}{\partial\theta},\frac{\partial\varphi}{\partial\phi}\right)
    \label{eqn:scalar one-form}
\end{equation}
This can also be expanded in the same way as before to find the matrix $\Lambda_{\bar\beta}^\alpha$ as the one-form transforms like 
\begin{equation*}
    \frac{\partial\varphi}{\partial x^{\bar\beta}}=\Lambda_{\bar\beta}^\alpha\frac{\partial\varphi}{\partial x^\alpha}
\end{equation*}
This gives us
\begin{equation}
    \Lambda_{\bar\beta}^\alpha=
    \begin{pmatrix}
        \sin\theta\cos\varphi&r\cos\theta\cos\varphi&-r\sin\theta\sin\varphi\\
        \sin\theta\sin\varphi&r\cos\theta\sin\varphi&r\sin\theta\cos\varphi\\
        \cos\theta&-r\sin\theta&0
    \end{pmatrix}
\end{equation}

We expect these two matrices to be related in the form $\Lambda_\beta^{\bar\alpha}\Lambda_{\bar\beta}^\alpha=\mathbbm{1}$, the identity matrix \cite{dunsby}. We can now also find the metric tensor for this coordinate system \cite{dunsby}:
\begin{equation}
    g_{\alpha\beta}=\Lambda^\gamma_{\bar\alpha}\Lambda^\delta_{\bar\beta}g_{\gamma\delta}
\end{equation}
In this case, $g_{\gamma\delta}$ is the metric for flat space. Note that we are not yet considering a time coordinate but at this point, with no mass in the system, it is identical to the time coordinate in Minkowski spacetime. We find our metric to be 
\begin{equation}
    g_{\bar\alpha\bar\beta}=
    \begin{pmatrix}
        1&0&0\\
        0&r^2&0\\
        0&0&r^2\sin^2\theta
    \end{pmatrix}
    \label{eqn:spherical polars metric}
\end{equation}
Since it is diagonal we can simply find the inverse metric by taking the inverse of each component:
\begin{equation}
    g^{\bar\alpha\bar\beta}=(g_{\bar\alpha\bar\beta})^{-1}=
    \begin{pmatrix}
        1&0&0\\
        0&\frac{1}{r^2}&0\\
        0&0&\frac{1}{r^2\sin^2\theta}
    \end{pmatrix}
    \label{eqn:spherical polars inverse metric}
\end{equation}

\section{The wave equation}\label{sec:Wave Equation}
\par The mechanism by which energy will be extracted from this black hole involves the passing of a scalar wave through a section of space near the black hole, called the ergosphere. The details of this will be fleshed out later but for now let us investigate the wave equation in these coordinates. In a general reference frame and coordinate system, the wave equation is given by
\begin{equation}
    g^{\alpha\beta}\nabla_\alpha\nabla_\beta \varphi=g^{\alpha\beta}\nabla_\alpha\frac{\partial \varphi}{\partial x^\beta}=0
    \label{eqn:general wave equation}
\end{equation}
where we are considering covariant derivatives. $\varphi$ is a scalar field, so its first derivative acts like a normal derivative. For convenience, we will define this first derivative as $\frac{\partial \varphi}{\partial x^\beta} = A_\beta$. The second derivative, however, needs to be dealt with properly, using the definition of the covariant derivative, from \cite{dunsby}:
\begin{equation}
    V_{\alpha;\beta}=V_{\alpha,\beta}+V_\mu\Gamma^\alpha_{\mu\beta}
    \label{eqn:covariant derivative}
\end{equation}
Our wave equation becomes
\begin{equation*}
    g^{\alpha\beta}\left(A_{\beta,\alpha}+A_\mu\Gamma^\alpha_{\mu\beta}\right)=0
\end{equation*}
and the problem has been reduced to simply finding the Christoffel symbols $\Gamma^\alpha_{\mu\beta}$. We will show the calculation of one of these for completeness, but the actual calculation was done with symbolic computing, using the \texttt{GRQUICK} module for Mathematica.
\par The Christoffel symbols are given by 
\begin{equation}
    \Gamma^\alpha_{\mu\beta}=\frac{1}{2}g^{\alpha\delta}[g_{\mu\delta,\beta}+g_{\beta\delta,\mu}-g_{\mu\beta,\delta}]
    \label{eqn:Christoffel symbol}
\end{equation}
So, choosing a combination of the indices that we know gives a non-zero result (thanks to Mathematica), we find
\begin{align*}
    \Gamma^r_{\phi\phi}&=\frac{1}{2}g^{r\delta}[g_{\phi\delta,\phi}+g_{\phi\delta,\phi}-g_{\phi\phi,\delta}]\\
    &=\frac{1}{2}g^{rr}[g_{\phi r,\phi}+g_{\phi r,\phi}-g_{\phi\phi,r}]
\end{align*}
since the only non-zero component of the inverse metric, in the $r$ row, is the $rr$ component, which is 1. We also notice that in the $\phi$ row of the metric, the only non-zero component is the $\phi\phi$ component, meaning the first two terms in the square bracket are zero.
\begin{align*}
    \therefore\Gamma^r_{\phi\phi}&=\frac{1}{2}[0+0-\frac{\partial}{\partial r}(r^2\sin^2\theta)]\\
    &=\frac{-1}{2}2r\sin^2\theta\\
    &=-r\sin^2\theta
\end{align*}
The other non-zero components were 
\begin{equation*}
    \Gamma^\phi_{\theta\phi}=\cot\theta,\;\;\;\Gamma^\phi_{r\phi}=\frac{1}{r},\;\;\;\Gamma^\theta_{\phi\phi}=-\sin\theta\cos\theta,\;\;\;\Gamma^\theta_{r\theta}=\frac{1}{r},\;\;\;\Gamma^r_{\theta\theta}=-r
\end{equation*}
Putting all these together, recalling our definition of $A_\beta$, and introducing the time component of the metric, which is simply -1, the wave equation becomes
\begin{align*}
    0&=g^{tt}\left(\frac{\partial^2\varphi}{\partial t^2}+\frac{\partial\varphi}{\partial x^\mu}\Gamma^t_{\mu t}\right)+g^{rr}\left(\frac{\partial^2\varphi}{\partial r^2}+\frac{\partial\varphi}{\partial x^\mu}\Gamma^r_{\mu r}\right)+g^{\theta\theta}\left(\frac{\partial^2\varphi}{\partial \theta^2}+\frac{\partial\varphi}{\partial x^\mu}\Gamma^\theta_{\mu \theta}\right)+g^{\phi\phi}\left(\frac{\partial^2\varphi}{\partial \phi^2}+\frac{\partial\varphi}{\partial x^\mu}\Gamma^\phi_{\mu \phi}\right)\\
    &=-(\varphi_{tt}+0)+(\varphi_{rr}+0)+\frac{1}{r^2}\left(\varphi_{\theta\theta}+\varphi_r\frac{1}{r}\right)+\frac{1}{r^2\sin^2\theta}\left(\varphi_{\phi\phi}+\varphi_r\frac{1}{r}+\varphi_\theta\cot\theta\right)
\end{align*}
Note that the metric and its inverse are diagonal, so we only get components from them when their indices are the same. Finally we have 
\begin{equation}
    -\varphi_{tt}+\varphi_{rr}+\frac{1}{r^2}\left(\varphi_{\theta\theta}+\varphi_r\frac{1}{r}\right)+\frac{1}{r^2\sin^2\theta}\left(\varphi_{\phi\phi}+\varphi_r\frac{1}{r}+\varphi_\theta\cot\theta\right)
    \label{eqn:wave equation spherical polars}
\end{equation}

\section{Simplifying the wave equation}\label{sec:simplifying wave equation}
\par \Cref{eqn:wave equation spherical polars} would work for our purposes, but there are a few things that we can do to simplify this calculation that actually remove the need to compute the Christoffel symbols entirely. We begin by defining the determinant of the metric as $\det||g_{\alpha\beta}||=g$. 
\par Now, for any diagonalisable matrix $A$, \cite{determinant thm} tells us that 
\begin{equation}
    \det(e^{A})=e^{\tr(A)}
    \label{eqn:determinant theorem}
\end{equation}
So, letting $B=\ln(A)$, we see
\begin{align*}
    \det(B)&=e^{\tr(\ln(B))}\\
    \implies \ln(\det(B))&=\tr(\ln(B))\\
    \implies \frac{\partial}{\partial x^\alpha}\ln(\det(B))&=\frac{\partial}{\partial x^\alpha}\tr(\ln(B))\\
    &=\tr(\frac{\partial}{\partial x^\alpha}\ln(B))\\
    &=\tr(B^{-1}\frac{partial}{\partial x^\alpha}B)
\end{align*}
Then using the fact that $\tr(A\cdot B)=A_{\mu\nu}B_{\mu\nu}$ and $(B_{\mu\nu})^{-1}=B^{\mu\nu}$ we have, substituting the metric in for B, our first identity:
\begin{equation}
    \frac{\partial}{\partial x^\alpha}\ln(g)=g^{\mu\nu}g_{\mu\nu,\alpha}
    \label{eqn:identity 1}
\end{equation}
\par Next up, we want to find a simplification of the Christoffel symbols. Recalling the definition of the Christoffel symbols and noting that in the wave equation we always get Christoffel symbols of the form $\Gamma^\alpha_{\alpha\beta}$, we can write
\begin{equation*}
    \Gamma^\alpha_{\alpha\beta}=\frac{1}{2}g^{\alpha\delta}[g_{\alpha\delta,\beta}+g_{\beta\delta,\alpha}-g_{\alpha\beta,\delta}]\\
\end{equation*}
Now notice that, if our metric is symmetric, all the off-diagonal terms will cancel with each other while the only contributions will come from the diagonal terms. In our case that means we only have non-zero terms where $\alpha=\delta$, so the second two terms end up cancelling, leaving
\begin{equation*}
    \Gamma^\alpha_{\alpha\beta}=\frac{1}{2}g^{\alpha\delta}g_{\alpha\delta,\beta}
\end{equation*}
Using \cref{eqn:identity 1} we can see that 
\begin{align*}
    \frac{1}{2}g^{\alpha\delta}g_{\alpha\delta,\beta}&=\frac{1}{2}\frac{\partial}{\partial x^\beta}\ln(g)\\
    &=\frac{\partial}{\partial x^\beta}\frac{1}{2}\ln(g)\\
    &=\frac{\partial}{\partial x^\beta}\ln(g^{1/2})
\end{align*}
Combining this we have a second identity:
\begin{equation}
    \Gamma^\alpha_{\alpha\beta}=\frac{\partial}{\partial x^\beta}\ln(\sqrt{|g|})
    \label{eqn:identity 2}
\end{equation}
Lastly, bringing it all back to the wave equation, we can write the wave equation as 
\begin{equation}
    g^{\alpha\beta}\nabla_\alpha\nabla_\beta \varphi=\nabla_\alpha\nabla^\alpha \varphi=\nabla_\alpha A^\alpha=0
    \label{eqn:wave equation A}
\end{equation}
where, similar to before, we have defined $A^\alpha = \nabla^\alpha \varphi$. The covariant derivative of a vector is given by \cref{eqn:covariant derivative} and switching that $\beta$ to an $\alpha$ we have the wave equation as given in \cref{eqn:wave equation A}:
\begin{align*}
    \nabla_\alpha A^\alpha&=A^\alpha_{,\alpha}+A^\mu\Gamma^{\alpha}_{\mu\alpha}\\
    &=A^\alpha_{,\alpha}+A^\mu\frac{\partial}{\partial x^\mu}\ln(\sqrt{|g|})\\
    &=A^\alpha_{,\alpha}+\frac{A^\mu}{\sqrt{|g|}}\frac{\partial}{\partial x^\mu}\sqrt{|g|}\\
    &=\frac{1}{\sqrt{|g|}}(\sqrt{|g|}A^\alpha)_{,\alpha}
\end{align*}
In the last line we have renamed the $\mu$ to $\alpha$ as it is a dummy index and then used product rule in reverse. Substituting the scalar field $\varphi$ back in, we have a new form of the wave equation:
\begin{equation}
    \frac{1}{\sqrt{|g|}}(\sqrt{|g|}\nabla^\alpha\varphi)_{,\alpha}=\frac{1}{\sqrt{|g|}}(\sqrt{|g|}g^{\alpha\beta}\nabla_\beta\varphi)_{,\alpha}=0
    \label{eqn:wave equation spherical polars final}
\end{equation}
This does not require the calculation of endless Christoffel symbols, merely the determinant of the metric, which is fairly easily done. Note that this \cref{eqn:wave equation spherical polars final} was derived using only the fact that the metric is diagonalisable and symmetric, which is a fairly common set of conditions for a variety of metrics.

\section{The Kerr metric}\label{sec:Kerr}
\par The class of black hole that we will be studying is a rotating, uncharged black hole. This is best described by the Kerr metric \cite{kerr}. We will also be using so-called geometrised units, where $G=c=1$, in order to simplify our calculations. The Kerr metric is defined as follows
\begin{equation}
    g_{\alpha\beta}=
    \begin{pmatrix}
        (-1+\frac{2mr}{\Sigma}) & 0 & 0 & \frac{-2amr\sin^2\theta}{\Sigma} \\
        0 & \frac{\Sigma}{\Delta} & 0 & 0 \\
        0 & 0 & \Sigma & 0 \\
        \frac{-2amr\sin^2\theta}{\Sigma} & 0 & 0 & \left(r^2+a^2+\frac{2mra^2\sin^2\theta}{\Sigma}\right)\sin^2\theta
    \end{pmatrix}
    \label{eqn:kerr metric}
\end{equation}
where we have defined $\Sigma=r^2+a^2\cos^2\theta$, $\Delta=r^2+a^2-2mr$, $m$ is the mass of the black hole, and $a$ is its angular momentum. 
\par The wave equation in \cref{eqn:wave equation spherical polars final} requires the form of the inverse metric, as well as the determinant of the metric. The inverse of \cref{eqn:kerr metric} is fairly simple to find. The $rr$ and $\theta\theta$ components can simply be inverted as they are independent of any other coordinate in the metric, and then the other 4 components can be found by treating them separately as a $2\times2$ matrix, for which the inverse is easily found. We find the inverse of \cref{eqn:kerr metric} to be
\begin{equation}
    g^{\alpha\beta}=
    \begin{pmatrix}
        -\frac{1}{\Delta}\left(r^2+a^2+\frac{2mra^2\sin^2\theta}{\Sigma}\right) & 0 & 0 & \frac{-2mra}{\Sigma\Delta} \\
        0 & \frac{\Delta}{\Sigma} & 0 & 0 \\
        0 & 0 & \frac{1}{\Sigma} & 0 \\
        \frac{-2mra}{\Sigma\Delta} & 0 & 0 & \frac{\Delta-a^2\sin^2\theta}{\Sigma\Delta\sin^2\theta}
    \end{pmatrix}
    \label{eqn:kerr metric inverse}
\end{equation}
\par We can check that this is in fact the inverse by multiplying \cref{eqn:kerr metric} by \cref{eqn:kerr metric inverse} and after a fair amount of algebra, which we will not get into here, it does return the identity matrix. 
\par We will also need the determinant of the Kerr-Newman metric. This calculation is fairly straightforward but tedious. We find it to be
\begin{equation}
    g=det(g_{\alpha\beta})=-\Sigma^2\sin^2\theta
\end{equation}

\section{The wave equation for the Kerr metric}\label{sec:kerr metric wave equation}
\par We now have the tools we need to tackle the wave equation. Recalling the form of the wave equation in \cref{eqn:wave equation spherical polars final}, we can expand the derivative by product rule to write
\begin{align*}
    0&=\frac{1}{\sqrt{|g|}}(\sqrt{|g|}g^{\alpha\beta}\nabla_\beta\varphi)_{,\alpha}\\
    &=\frac{1}{\sqrt{|g|}}\sqrt{|g|}g^{\alpha\beta}\varphi_{,\beta\alpha}+\frac{1}{\sqrt{|g|}}\sqrt{|g|}g^{\alpha\beta}_{,\alpha}\varphi_{,\beta}+\frac{1}{\sqrt{|g|}}\frac{-1}{2\sqrt{|g|}}g_{,\alpha}g^{\alpha\beta}\varphi_{,\beta}\\
    &=g^{\alpha\beta}\varphi_{,\beta\alpha}+g^{\alpha\beta}_{,\alpha}\varphi_{,\beta}-\frac{1}{2|g|}g_{,\alpha}g^{\alpha\beta}\varphi_{,\beta}
\end{align*}
\par With this, we can formulate the wave equation for the Kerr system. We begin by noting some useful things: The components of the inverse metric depend only on $r$ and $\theta$, so any derivatives of the inverse metric with respect to $t$ or $\phi$ vanish. For this reason, the second term in the equation above is non-zero only for $\alpha=\beta=(r,\theta)$, so there are only derivatives of $\varphi$ with respect to $r$ and $\theta$ in that term. The same is true for the determinant of the metric, resulting in the gradient of the determinant only having $r$ and $\theta$ terms. This results in a similar simplification of the third term. Let us expand the wave equation considering these simplifications:
\begin{align*}
    0=&\left[g^{tt}\frac{\partial^2\varphi}{\partial t^2}+g^{rr}\frac{\partial^2\varphi}{\partial r^2}+g^{\theta\theta}\frac{\partial^2\varphi}{\partial\theta^2}+g^{\phi\phi}\frac{\partial^2\varphi}{\partial\phi^2}-g^{\phi t}\frac{\partial^2\varphi}{\partial t\partial\phi}-g^{t\phi}\frac{\partial^2\varphi}{\partial\phi\partial t}\right]\\&+\left[\frac{\partial g^{rr}}{\partial r}\frac{\partial\varphi}{\partial r}+\frac{\partial g^{\theta\theta}}{\partial \theta}\frac{\partial \varphi}{\partial \theta}\right]-\frac{1}{2|g|}\left[g_{,r}g^{rr}\frac{\partial\varphi}{\partial r}+g_{,\theta}g^{\theta\theta}\frac{\partial\varphi}{\partial\theta}\right]
\end{align*}
Calculating the terms we are missing is fairly simple:
\begin{align*}
    \frac{\partial g^{rr}}{\partial r}&=\frac{\partial}{\partial r}\frac{\Delta}{\Sigma}\\
    &=\frac{(2r-2m)\Sigma-\Delta(2r)}{\Sigma^2}\\
    \frac{\partial g^{\theta\theta}}{\partial \theta}&=\frac{\partial}{\partial \theta}\frac{1}{\Sigma}\\
    &=\frac{-2a^2\cos\theta\sin\theta}{\Sigma^2}\\
    g_{,r}&=\frac{\partial}{\partial r}(-\Sigma^2\sin^2\theta)\\
    &=-4r\Sigma\sin^2\theta\\
    g_{,\theta}&=\frac{\partial}{\partial \theta}(-\Sigma^2\sin^2\theta)\\
    &=4\Sigma\sin^3\theta a^2\cos\theta-\Sigma^2 2\sin\theta\cos\theta
\end{align*}
And finally we can substitute everything into the wave equation:
\begin{align*}
    0=&\left[-\frac{1}{\Delta}\left(r^2+a^2+\frac{2mra^2\sin^2\theta}{\Sigma}\right)\frac{\partial^2\varphi}{\partial t^2}+\frac{\Delta}{\Sigma}\frac{\partial^2\varphi}{\partial r^2}+\frac{1}{\Sigma}\frac{\partial^2\varphi}{\partial\theta^2}+\frac{\Delta-a^2\sin^2\theta}{\Sigma\Delta\sin^2\theta}\frac{\partial^2\varphi}{\partial\phi^2}-\frac{4mra}{\Sigma\Delta}\frac{\partial^2\varphi}{\partial t\partial\phi}\right]\\
    &+\left[\frac{(2r-2m)\Sigma-\Delta(2r)}{\Sigma^2}\frac{\partial\varphi}{\partial r}+\frac{-2a^2\cos\theta\sin\theta}{\Sigma^2}\frac{\partial \varphi}{\partial \theta}\right]\\&-\frac{1}{2\Sigma^2\sin^2\theta}\left[(-4r\Sigma\sin^2\theta)\frac{\Delta}{\Sigma}\frac{\partial\varphi}{\partial r}+(4\Sigma\sin^3\theta a^2\cos\theta-\Sigma^2 2\sin\theta\cos\theta)\frac{1}{\Sigma}\frac{\partial\varphi}{\partial\theta}\right]
\end{align*}
We can multiply this whole equation by $\Sigma$ and then find that the second and third term cancel most of their components. We find
\begin{equation}
    \begin{aligned}
        0=&\left(-\frac{(r^2+a^2)^2}{\Delta}+a^2\sin^2\theta\right)\frac{\partial^2\varphi}{\partial t^2}+\frac{\partial}{\partial r}\left(\Delta\frac{\partial\varphi}{\partial r}\right)+\frac{1}{\sin\theta}\frac{\partial}{\partial\theta}\left(\sin\theta\frac{\partial\varphi}{\partial\theta}\right)\\&+\left(\frac{1}{\sin^2\theta}-\frac{a^2}{\Delta}\right)\frac{\partial^2\varphi}{\partial\phi^2}-\frac{4mra}{\Delta}\frac{\partial^2\varphi}{\partial t\partial\phi}
    \end{aligned}
    \label{eqn:wave equation kerr metric}
\end{equation}
The most important things to note when getting to this result are noticing the product rule terms for the derivatives with respect to $r$ and to $\theta$, as well as manipulating the term multiplying the time derivative into the right form. 

\section{Solving the wave equation}\label{sec:Solving the wave equation}
\par Now that we have the form of the wave equation in spacetime containing our rotating, uncharged black hole, we can begin to investigate the properties of waves in this spacetime. In order to do this, we will use separation of variables. We assume the scalar field can be represented as a product of components which each independently depend on a single coordinate:
\begin{equation}
    \varphi(t,r,\theta,\phi)=e^{-i\omega t}e^{im\phi}R(r)S(\theta)
    \label{eqn:sep of variables ansatz}
\end{equation}
We have assumed the form of the $t$- and $\phi$-dependent parts to simply be waves as, by inspection of \cref{eqn:wave equation kerr metric} we can see that derivatives of $\varphi$ with respect to those components are always double derivatives. For the $r$ and $\theta$ derivatives, there is some product rule stuff going on, so we cannot assume their forms. Substituting this ansatz into \cref{eqn:wave equation kerr metric}, we can begin to simplify things:
\begin{align*}
    0=&\left(-\frac{(r^2+a^2)^2}{\Delta}+a^2\sin^2\theta\right)(-\omega^2)(e^{-i\omega t}e^{im\phi}R(r)S(\theta))+\frac{\partial}{\partial r}\left(\Delta\frac{\partial R(r)}{\partial r}\right)(e^{-i\omega t}e^{im\phi}S(\theta))\\&+\frac{1}{\sin\theta}\frac{\partial}{\partial\theta}\left(\sin\theta\frac{\partial S(\theta)}{\partial\theta}\right)(e^{-i\omega t}e^{im\phi}R(r))+\left(\frac{1}{\sin^2\theta}-\frac{a^2}{\Delta}\right)(-m^2)(e^{-i\omega t}e^{im\phi}R(r)S(\theta))\\&-\frac{4mra}{\Delta}(m\omega)(e^{-i\omega t}e^{im\phi}R(r)S(\theta))
\end{align*}
Assuming $R(r)$ and $S(\theta)$ are non-zero as that would be the trivial solution, we can divide through by $e^{-i\omega t}e^{im\phi}R(r)S(\theta)$ and then separate all terms with $r$ dependence from terms with $\theta$ dependence:
\begin{equation*}
    \frac{1}{R}\frac{\partial}{\partial r}\left(\Delta\frac{\partial R}{\partial r}\right)+\frac{\omega^2(r^2+a^2)^2}{\Delta}+\frac{m^2a^2}{\Delta}-\frac{4m^2ra\omega}{\Delta}=\omega^2a^2\sin^2\theta-\frac{1}{S\sin\theta}\frac{\partial}{\partial\theta}\left(\sin\theta\frac{\partial S(\theta)}{\partial\theta}\right)+\frac{m^2}{\sin^2\theta}
\end{equation*}
\par We are looking to investigate the changes of energy of the scalar waves as they travel near the black hole. $\theta$ is the coordinate which represents the azimuthal angle relative to the origin. The origin that makes the most sense in our case is the centre of the black hole, so $\theta$ simply represents moving from the ``north pole'' to the ``south pole'' at a fixed radius. This shouldn't change the potential energy and thus is of no interest to us. We will focus on the radial part of the above equation. Without loss of generality, we can equate it to some constant, which we will call $A\in\mathbb{R}$.
\begin{equation}
    \frac{1}{R}\frac{\partial}{\partial r}\left(\Delta\frac{\partial R}{\partial r}\right)+\frac{\omega^2(r^2+a^2)^2+m^2a^2-4m^2ra\omega}{\Delta}=A
    \label{eqn:radial wave equation}
\end{equation}
To simplify further, we will introduce a change of variable, called the tortoise coordinate $r^*$ such that
\begin{equation}
    \frac{dr^*}{dr}=\frac{r^2+a^2}{\Delta}
    \label{eqn:tortoise coordinates}
\end{equation}
\Cref{eqn:radial wave equation} becomes
\begin{align*}
    0&=\frac{\partial}{\partial r}\left(\Delta\frac{\partial R}{\partial r}\right)+\left[\frac{\omega^2(r^2+a^2)^2+m^2a^2-4m^2ra\omega}{\Delta}-A\right]R\\
    &=\frac{\partial r^*}{\partial r}\frac{\partial}{\partial r^*}\left(\Delta\frac{\partial R}{\partial r^*}\frac{\partial r^*}{\partial r}\right)+\left[\frac{\omega^2(r^2+a^2)^2+m^2a^2-4m^2ra\omega}{\Delta}-A\right]R\\
    &=\frac{r^2+a^2}{\Delta}\frac{\partial}{\partial r^*}\left(\Delta\frac{r^2+a^2}{\Delta}\frac{\partial R}{\partial r^*}\right)+\left[\frac{\omega^2(r^2+a^2)^2+m^2a^2-4m^2ra\omega}{\Delta}-A\right]R\\
    &=\frac{(r^2+a^2)^2}{\Delta}\frac{\partial^2 R}{\partial {r^*}^2}+2r\frac{\partial R}{\partial r^*}+\left[\frac{\omega^2(r^2+a^2)^2+m^2a^2-4m^2ra\omega}{\Delta}-A\right]R\\
    &=\frac{\partial^2 R}{\partial {r^*}^2}+\frac{2r\Delta}{(r^2+a^2)^2}\frac{\partial R}{\partial r^*}+\left[\omega^2+\frac{m^2a^2-4m^2ra\omega-A\Delta}{(r^2+a^2)^2}\right]R
\end{align*}
\par The region of space around the black hole that is of interest to us is called the ergosphere. It is a shell around the black hole, characterised by two radii. These radii are actually already known to us, as the roots of the equation $\Delta=0$:
\begin{align*}
    \Delta=0&=r^2+a^2-2mr\\
    \implies r_\pm&=m\pm\sqrt{m^2-a^2}
\end{align*}
The outer radius, $r_+$, is the one of interest to us. We want to examine the change in energy of the wave from when it enters the ergosphere to when it leaves. At this $r_+$, $\Delta=0$ and we can write $r_+^2+a^2=2mr_+$, so the radial wave equation becomes
\begin{align*}    
    0&=\frac{\partial^2 R}{\partial {r^*}^2}+\left[\omega^2+\frac{m^2a^2}{(2mr_+)^2}-\frac{4m^2r_+a\omega}{(2mr_+)^2}\right]R\\
    &=\frac{\partial^2 R}{\partial {r^*}^2}+\left[\omega^2+\frac{m^2a^2}{(2mr_+)^2}-\frac{2ma\omega}{2mr_+}\right]R\\
    &=\frac{\partial^2 R}{\partial {r^*}^2}+\left[\omega-\frac{ma}{2mr_+}\right]^2R
\end{align*}
This is a simple ODE that can be solved by an exponential. Defining $\omega_\pm=\frac{a}{2mr_\pm}$, we can write our solution to the above equation as
\begin{equation}
    R=e^{\pm i(\omega-m\omega_+)r^*}
    \label{eqn:radial wave equation solution}
\end{equation}
This is clearly some sort of wave, which we can add back to our solution of \cref{eqn:wave equation kerr metric}:
\begin{equation}
    \varphi=e^{-i\omega t}e^{im\phi}e^{\pm i(\omega-m\omega_+)r^*}S(\theta)
    \label{eqn:scalar field}
\end{equation}
Once again, the form of the $\theta$ component is not of interest to us. 

\section{Finding the Energy of These Waves}
\par We can now examine the energy of this scalar field. To do this, we use the energy momentum tensor for a scalar field, as derived in \cite[eqn. 4]{energy momentum}. Lowering the indices gives us
\begin{equation}
    T_{\alpha\beta}=\varphi_{,\alpha}\varphi_{,\beta}-\frac{1}{2}g_{\alpha\beta}(\varphi_{,\gamma}\varphi^{,\gamma})
    \label{eqn:energy momentum}
\end{equation}
We can also define the energy flux vector:
\begin{equation}
    4\pi J^\alpha=T_t^\alpha=g^{\alpha\delta}T_{t\delta}
    \label{eqn:energy flux}
\end{equation}
We are interested in the energy flux through the boundary of the ergosphere, in the radial direction. So we examine $J^r$:
\begin{align*}
    J^r&=g^{r\delta}T_{t\delta}\\
    &=g^{rr}T_{t,r}\\
    &=\frac{\Delta}{\Sigma}T_{t,r}
\end{align*}
In the second line we have taken the fact that the only non-zero component of the inverse metric in the $r$ row is $g^{rr}$. The component $T_{t,r}$ can be computed:
\begin{align*}
    T_{t,r}&=\varphi_{,t}\varphi_{,r}-\frac{1}{2}g_{tr}(\varphi_{,\gamma}\varphi^{,\gamma})\\
    &=\varphi_{,t}\varphi_{,r}\\
    &=(-i\omega)\varphi(\varphi_{,r^*}\frac{dr^*}{dr})\\
    &=(-i\omega)\varphi\left(\pm i(\omega-m\omega_+)\frac{r^2+a^2}{\Delta}\right)\varphi
\end{align*}
So we can express the energy flux vector as
\begin{equation*}
    4\pi J^r=\pm\omega(\omega-m\omega_+)\frac{r^2+a^2}{\Sigma}\varphi^2
\end{equation*}
The real part of the energy flux vector is the only part that resembles a physical quantity, so taking the real part of the equation above we find
\begin{equation*}
    \mathfrak{Re}(4\pi J^r)=\pm\omega(\omega-m\omega_+)\frac{r^2+a^2}{\Sigma}S(\theta)^2
\end{equation*}
If we consider this energy flux at the outer radius of the ergosphere, $r_+$, then we have 
\begin{equation}
    \mathfrak{Re}(4\pi J^r)=\pm\omega(\omega-m\omega_+)\frac{2mr_+}{\Sigma}S(\theta)^2
    \label{eqn:energy flux real}
\end{equation}
The $\pm$ in \cref{eqn:energy flux real} represents the wave travelling into $(-)$ or out of $(+)$ the ergosphere.
\par Finally, we can compute the change of total energy within the outer shell of the ergosphere:
\begin{equation*}
    \frac{dE}{dt}=\int T_0^r|g|d\theta d\phi
\end{equation*}
This $|g|$ is the determinant of the metric for the two-sphere, which is simply $\sin\theta$. We will consider a wave exiting the ergosphere, resulting in 
\begin{align*}
    \frac{dE}{dt}&=\frac{1}{4\pi}\omega(\omega-m\omega_+)\frac{2mr_+}{\Sigma}\int S(\theta)^2\sin\theta d\theta d\phi\\
    &=\omega(\omega-m\omega_+)\frac{mr_+}{\Sigma}\int S(\theta)^2\sin\theta d\theta
\end{align*}
This value is positive only if $(\omega-m\omega_+)$ is positive, as the rest of the terms are always positive. So, we will see the energy inside the sphere decrease only if $\omega-m\omega_+<0$. In other words, recalling the definition of $\omega_+$, if 
\begin{equation}
    \frac{a}{2r_+}>\omega
    \label{eqn:condition for energy decrease}
\end{equation}
This is the condition for a scalar wave travelling through the ergosphere to carry energy out as it exits the ergosphere at the outer radius. 

\section{Conclusion}\label{sec:Conclusion}
\par We began by investigating the wave equation in spherical polar coordinates, and then simplifying the calculation with some useful identities. From this, we were able to apply the same simplifications when considering the spacetime around a rotating, uncharged black hole, described by the Kerr metric. Solving the radial part of the wave equation and considering a wave travelling out of the ergosphere of the black hole, we were able to see that the wave would gain energy as it leaves, provided its frequency obeys the relation in \cref{eqn:condition for energy decrease}. 

\section*{Acknowledgements}
This work was supported by Professor Peter Dunsby as part of the Integrated Assessment for 3rd year Applied Maths at the University of Cape Town.


\begin{thebibliography}{9}
    \bibitem{Penrose}
    Penrose, R. and Floyd, R. (1971). `Extraction of Rotational Energy from a Black Hole'. \textit{Nature Physical Science}, 229(6), 177-179. \url{https://doi.org/10.1038/physci229177a0}

    \bibitem{kerr}
    Kerr, R. (1963). `Gravitational Field of a Spinning Mass as an Example of Algebraically Special Metrics'. \textit{Physical Review Letters}, 11(5), 237-238. \url{https://doi.org/10.1103/physrevlett.11.237}
    
    \bibitem{dunsby}
    Dunsby, P. K. S. (2015). \textit{An Introduction to Tensors and Relativity}, University of Cape Town

    \bibitem{determinant thm}
    A Beautiful Theorem Concerning Determinants. (2021). \url{http://applet-magic.com/determinanttheorem.htm}

    \bibitem{energy momentum}
    hep.itp.tuwien.ac.at. (2021). Retrieved 4 November 2021, from \url{http://hep.itp.tuwien.ac.at/~wrasetm/files/2017S-GRplusScalar.pdf}.
\end{thebibliography}
    
\end{document}
\documentclass[12pt]{article}
\usepackage[margin=1.2in]{geometry}
\usepackage{graphicx}
\usepackage{amsmath}
\usepackage{physics}
\usepackage{tabto}
\usepackage{float}
\usepackage{amssymb}
\usepackage{pgfplots}
\usepackage{verbatim}
\usepackage{tcolorbox}
\usepackage{listings}
\usepackage{xcolor}
\usepackage{siunitx}
\usepackage[all]{nowidow}
\definecolor{stringcolor}{HTML}{C792EA}
\definecolor{codeblue}{HTML}{2162DB}
\definecolor{commentcolor}{HTML}{4A6E46}
\lstdefinestyle{appendix}{
    basicstyle=\ttfamily\footnotesize,
    commentstyle=\color{commentcolor},
    keywordstyle=\color{codeblue},
    stringstyle=\color{stringcolor},
    showstringspaces=false,
    numbers=left,
    upquote=true,
    captionpos=t,
    abovecaptionskip=12pt,
    belowcaptionskip=12pt,
    language=Python,
    breaklines=true,
    frame=single,
}
\renewcommand{\lstlistingname}{Appendix}
\pgfplotsset{compat=1.16}

\title{Car on a Washboard Road Surface}
\date{\textbf{25 March 2020}}
\author{}

\begin{document}

    \maketitle
    \center
    \textbf{\large{MAM2046W 2OD}}\\
    \textbf{\large{KDSMIL001}}\\

    \begin{enumerate}
        \item \textbf{Modelling} \newline
        In order to model this system, we must consider all of the forces acting on the car. 
        It's relatively safe to assume that the car is in an inertial reference frame, therefore 
        we know that $\vec{F}_{net}=m\vec{a}$. On the other hand, we know that the only forces 
        acting on the car are the force of gravity $\vec{F}_G = m\vec{g}$ and the force of the 
        "spring", which can be modelled as $\vec{F}_S = k\Delta \vec{y}$ where $\Delta \vec{y}$ 
        is the distance from the equilibrium position to the current position of the mass.
        Now if we consider the system when it's at rest, in other words when the spring is at a 
        relative equilibrium position, there is a force being applied on the car by the spring in 
        order to perfectly balance the force of gravity, in which case we can effectively ignore 
        the force of gravity and choose the new position from which to measure $\Delta \vec{y}$, 
        leaving us with

        \begin{equation}
            \vec{F}_{net} = k\Delta \vec{y}
        \end{equation}

        \noindent
        Now we must consider the dashpot, which applies a force on the mass in proportion to 
        the velocity of the mass in the form $\vec{F}_D = c\vec{v}$. Adding this into our 
        equation for $\vec{F_{net}}$, we find 

        \begin{equation}
            \vec{F}_{net} = k\Delta \vec{y} + c\vec{v}
        \end{equation}

        \noindent
        With a usual mass on a spring system, this is as far as it goes as the only thing that 
        moves is the mass, but in this case both the mass and the connection point of the 
        "spring" are moving and they're not necessarily moving in sync with each other. In 
        order to account for this we need to modify the $\Delta\vec{y}$ and $\vec{v}$ terms 
        as they will not be changing in a simple manner. For the $\Delta\vec{y}$ term, this 
        isn't too hard to do. We just need to consider the effect that different values of 
        $\Delta\vec{y}$ will have on $\vec{F}_S$. From this we find 

        \begin{equation}
            \vec{F}_S = k(Y(t) - y(t))
        \end{equation}

        \noindent
        where $Y(t)$ is the upward displacement of the car and $y(t)$ is the upward displacement 
        of the connection point of the "spring", given by 

        \begin{equation}
            \begin{split}
                y(t) &= a\sin\frac{2\pi x}{\lambda} \\
                &= a\sin\frac{2\pi v t}{\lambda}
            \end{split}
        \end{equation}

        \noindent
        as $x = vt$.
    \end{enumerate}

\end{document}
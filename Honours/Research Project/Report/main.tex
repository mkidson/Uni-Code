% region preamble
\documentclass[11pt]{article}
\usepackage{import}
\usepackage[margin=1in, top=1in]{geometry}
\usepackage[all]{nowidow}
\usepackage[hyperfigures=true, hidelinks, pdfhighlight=/N]{hyperref}
\usepackage[separate-uncertainty=true, group-digits=false]{siunitx}
\usepackage{graphicx,amsmath,physics,tabto,float,amssymb,pgfplots,verbatim,tcolorbox}
\usepackage{listings,xcolor,subcaption,caption,import,wrapfig,lipsum,tikz,biblatex,dirtree}
\usepackage{minted,soul}
\usepackage[version=4]{mhchem}
\usepackage[noabbrev]{cleveref}
\newcommand{\creflastconjunction}{, and\nobreakspace}
\newcommand{\mb}[1]{\mathbf{#1}}
\newcommand{\pt}{p_\mathrm{T}}
\newcommand{\Cppinline}[1]{\mintinline{c++}{#1}}
\numberwithin{equation}{section}
\numberwithin{figure}{section}
\numberwithin{table}{section}
\definecolor{stringcolor}{HTML}{C792EA}
\definecolor{codeblue}{HTML}{2162DB}
\definecolor{commentcolor}{HTML}{4A6E46}
\definecolor{LightGray}{HTML}{eaeaea}
\captionsetup{font=small, belowskip=0pt}
\usemintedstyle{vs}
\setminted{framesep=2mm,bgcolor=LightGray,fontsize=\footnotesize,linenos,breaklines}
\let\OldTexttt\texttt
\sethlcolor{LightGray}
\renewcommand{\texttt}[1]{\OldTexttt{\hl{#1}}}
\renewcommand{\lstlistingname}{Appendix}
\pgfplotsset{compat=1.17}
\addbibresource{bibliography.bib}
% endregion

\title{{\Huge A preliminary analysis of data from ALICE's new ITS and MFT detectors}}
\author{{\Large Miles Kidson}\\ \\
Supervisors: Prof. Zinhle Buthelezi, Dr. SV Fortsch, \& Prof. Tom Dietel\\
Assisted By: Dr. B Naik (Postdoctoral fellow)}
\date{\textbf{UCT Honours 2022}}

\begin{document}
\pagenumbering{roman}
\maketitle

\begin{figure}[h]
    \begin{center}
        \includegraphics{Figs/UCT.jpg}
    \end{center}
\end{figure}

\begin{abstract}
    \centering
    We investigate pilot-beam data from proton-proton collisions in Run 3 at ALICE, tracing its journey through two of the new detectors (the Inner Tracking System and Muon Forward Tracker) and the new Online-Offline analysis framework. 
\end{abstract}

\newpage
\tableofcontents

\newpage
\listoffigures

\newpage
\pagenumbering{arabic}
\setcounter{page}{1}
\section{Introduction}\label{sec:Introduction}
The ALICE (A Large Ion Collider Experiment) detector is a detector experiment at the Large Hadron Collider (LHC) at CERN. Its primary goal is the investigation of ``strongly interacting matter at extreme energy densities, where a formation of a new phase of matter, the quark-gluon plasma, is expected''~\cite{ALICE_LOI}. It achieves this goal by studying the products of head-on collisions of heavy ions such as lead, called Pb-Pb collisions for short. It also studies proton-lead (p-Pb) and proton-proton (p-p) collisions. 

Run 3 is the latest period of data capture at the LHC, with an intended centre of mass energy per collision of $\sqrt{s}=\SI{13.6}{\tera\electronvolt}$ and increased luminosity of collisions---a factor 10 increase in integrated luminosity for Pb-Pb collisions. For Run 3, ALICE is moving from a triggered read-out system to a combination of triggered and continuous read-out. In order to achieve this, many detectors and their front-end electronics were upgraded, some new detectors were added, and the analysis framework was overhauled entirely. 

Of the upgrades to ALICE, the following are the subject of this report. The Inner Tracking System (ITS) was upgraded with an entirely new pixel detector technology, hoping to greatly increase the resolution when determining the primary collision vertex. The Muon Forward Tracker (MFT) is one of the new detectors added. Its primary use is to assist the Muon Spectrometer (MCH) with vertexing and tracking in the forward region of ALICE and uses the same technology as the ITS. To deal with the increased volumes of data, a new analysis framework was introduced called Online-Offline (O2). 

This report aims to introduce the detectors and the data they output, trace the path of that data through the new analysis framework until it arrives at our desk, and show the details of the analysis we performed on it. The data used in this report is from two proton-proton collision runs performed in October 2021, at a centre-of-mass energy of \SI{900}{\giga\electronvolt}. This is not an energy we expect to use for physics data analysis but is good enough for this purpose.

\section{Background \& Detector Theory}\label{sec:Background_Detector}
\subsection{Coordinates}
The coordinate system used at ALICE needs to be discussed in order to fully explain the scope of this report. A modified cylindrical coordinate system, shown in \cref{fig:coords}, is used as most detectors in the experiment are cylindrically symmetric about the beamline of the LHC. 

\begin{figure}[h]
    \begin{center}
        \includegraphics[width=.8\textwidth]{Figs/coords.pdf}
        \caption{Modified cylindrical coordinate system used at the LHC~\cite{coords}.}
        \label{fig:coords}
    \end{center}
\end{figure}

We place the $z$-axis along the beamline with its origin at the interaction point (IP). The IP is the point at which collisions happen, right in the center of the detector. The angle around the $z$-axis is called the azimuthal angle, denoted by $\varphi$. Sometimes in the literature $\varphi$ ranges from 0 to $2\pi$ and sometimes it ranges from $-\pi$ to $\pi$. We will try stay consistent and use the latter in this report, but we may need use the other convention at times. The angle from the $z$-axis to the $x$-$y$ plane is called the polar angle, denoted by $\theta$, and runs from 0 to $\pi$. We are interested in the standard 3-momentum of particles that we track in the detector, which we call $\vec{p}=(p_x,p_y,p_z)$, but we also define the transverse momentum as 
\begin{equation}
    p_{\mathrm{T}}=\sqrt{p_x^2 + p_y^2}.
    \label{eqn:transverse momentum}
\end{equation}
We define the rapidity, often denoted as $y$, as
\begin{equation}
    y=\frac 12 \ln\left(\frac{E+p_z}{E-p_z}\right)
    \label{eqn:rapidity}
\end{equation}
where $E$ is the total energy of the particle being considered and $p_z$ is the momentum in the $z$ direction~\cite{kar_exp_phys}. This quantity is useful as differences in rapidity are Lorentz invariant for boosts along the $z$-axis. One issue, however, is that the energy of a particle is hard to measure, so we instead use pseudorapidity, denoted as $\eta$. Rapidity and pseudorapidity are equivalent for massless particles, and near equivalent for particles with total 3-momentum magnitude $p$ much greater than their mass $m$. Pseudorapidity is much easier to measure as it is defined in \cite{kar_exp_phys} as
\begin{equation}
    \eta=-\ln\tan\frac{\theta}{2}.
    \label{eqn:pseudorapidity}
\end{equation}
From \cref{fig:coords} we see that for $z$ positive, $\eta$ is also positive, and similarly for $z$ negative. Confusingly, we define the ``forward region'' of the ALICE detector as the region for which $z$, and thus $\eta$, are negative. The forward region is where our interest lies.

The four coordinates that we use most often are $z$, $\varphi$, $p_\mathrm{T}$, and $\eta$. 

\subsection{ALICE Run 3}
In 2018, the LHC shut down for what was called Long Shutdown 2 (LS2). During this time, the ALICE experiment was being prepared for Run 3, where it will be taking data at higher energies and much higher luminosities than before, from 2022 until 2025~\cite{ALICE_Upgrade_LOI}. \Cref{fig:ALICE_Schematic} shows the detector configuration for Run 3. The intent of these upgrades was in large part to prepare ALICE for a higher luminosity of collisions in both Pb-Pb and p-p cases. 

Part of the upgrades for Run 3, the details of which can be found in \cite{ALICE_Upgrade_LOI}, were a whole new Inner Tracking System and a brand new detector called the Muon Forward Tracker. These detectors are both silicon-based and their primary purpose is tracking particles and determining the collision vertex, which is the best estimation of where the collision that resulted in these particles happened. 

The readout electronics for many detectors were upgraded to allow for continuous readout where necessary. The MCH also had its readout and front-end electronics upgraded but will still work on a triggered readout system.

\begin{figure}[h]
    \begin{center}
        \includegraphics[width=\textwidth]{Figs/ALICE_RUN3_schematic.png}
        \caption{Schematic view of the ALICE detector setup for Run 3 of the LHC~\cite{ALICE_schematic_labels}. Note here that the MCH is shown separate from the MID, which act as the triggering mechanism for the MCH. For the purposes of this report, the MID will be considered part of the MCH. The ITS (6, 7), MCH (8), and MFT (9) are the focus of this report.}
        \label{fig:ALICE_Schematic}
    \end{center}
\end{figure}


\subsection{The Inner Tracking System}
The Inner Tracking System (ITS) sits in the main barrel of ALICE, as seen in \cref{fig:ALICE_Schematic}, and covers the range $|\eta|<1.22$~\cite{ITS_Upgrade_TDR}. For Run 3 it has been upgraded significantly by replacing the old detector with a new layout and new pixel detector technology, leading to an improvement in track position resolution at the primary vertex of a factor of 3 or greater~\cite{ITS_Upgrade_TDR}. The ITS's main purpose is to track the particles resulting from the collisions and determine the position of the primary vertex of collisions. It also serves to ``reconstruct secondary vertices, track and identify particles with low momentum, and improve the momentum and angle resolution for particles reconstructed by the Time Projection Chamber (TPC)''~\cite{ITS_Info}.

\begin{figure}[h]
    \begin{center}
        \includegraphics[width=.8\textwidth]{Figs/ITS_Schematic.png}
        \caption{Schematic view of the Inner Tracking System~\cite{ITS_Upgrade_TDR}. Note the thinner beam pipe and extremely close Inner Barrel.}
        \label{fig:ITS_Schematic}
    \end{center}
\end{figure}

The new ITS consists of 7 layers of pixel detectors; 3 in the ``Inner Barrel'' and 4 in the ``Outer Barrel''. The innermost layer sits at a radius of only \SI{22.4}{\milli\metre} from the IP thanks to a reduction in beam pipe radius for Run 3 and the outermost layer sits at a radius of \SI{391.8}{\milli\metre} from the IP. \Cref{fig:ITS_Schematic} shows the layout more clearly. 

The pixel detectors used are \SI{0.18}{\micro\metre} CMOS chips from TowerJazz. When a charged particle passes through the silicon in the active volume, it liberates the charge carriers in the material, which then collect on electrodes connected to the silicon, telling the detector that a particle has been detected. The fine segmentation of the detectors also allows the detector to determine the point at which the particle hit the detector, up to a resolution of \SI{4}{\micro\metre} in both the $r\varphi$ and $z$ directions~\cite{ITS_Upgrade_TDR}. The amount of charge deposited on the detector is dependent on the particle species and momentum (Bethe-Bloch). 

Two main methods of readout were considered for the ITS in the new continuous readout scheme. Firstly, a rolling shutter which continually loops through the rows of pixels and reads out the charge deposited on that pixel in the time since the last readout was considered. The time between readings, known as the integration time, for the first method is around \SI{30}{\micro\second}. The rolling shutter scheme lends itself to needing a small number of transistors within each pixel. The second scheme is known as ALPIDE, where each pixel has a comparator that signals when the pixel has an analogue signal greater than the comparator's threshold. The signalled pixels then get read out asynchronously, according to their priority in the chain. This scheme has an effective integration time of around \SI{4}{\micro\second} but has a larger material budget. 

\textbf{\textit{FIND OUT WHICH ONE WAS USED SMH}}




\subsection{The Muon Spectrometer}
The MCH sits in the forward region of ALICE, as seen in \cref{fig:ALICE_Schematic}, and covers $-4<\eta<-2.5$. It is designed to study heavy quark resonances through their single- and di-muon decay channels. As is shown in \cref{fig:Muon Spectrometer}, it is composed of a hadronic absorber, 5 tracking chambers, a dipole magnet, another absorber, and finally the 2 trigger chambers. The MFT is often also considered part of the MCH but is not shown in \cref{fig:Muon Spectrometer}. The section is adapted from the MCH Technical Design Report~\cite{MCH_TDR}.

\begin{figure}[ht]
    \begin{center}
        \includegraphics[width=0.8\textwidth]{Figs/MCH_schematic_pog.png}
        \caption{Diagram of the layout of the Muon Spectrometer~\cite{Muon_Spec_Schematic}. Muons pass through the absorber, are deflected by the dipole magnet, and hit the trigger chambers at the back. Importantly, all detector material sits behind the hadronic absorber, meaning the amount of data that can be used for tracking and vertexing is much lower than, say, the ITS.}
        \label{fig:Muon Spectrometer}
    \end{center}
\end{figure}

Muons don't tend to interact with matter much, especially when compared to hadrons and electrons. This makes studying muons easier, to some degree, since most other particles can be filtered out by making them pass through a large chunk of matter. As is shown in \cref{fig:Muon Spectrometer}, that is exactly what is done in the MCH. In front of any detector material (ignoring the MFT for now) sits the hadronic absorber, made primarily of carbon and concrete. This is intended to filter out all non-muon particles (mainly hadrons and photons) while not reducing the muon energy much so that they can still be studied. 

After the absorber are 5 sets of 2 cathode pad tracking detectors, situated around a large warm dipole magnet. Particles that make it through the absorber get picked up by the first two sets of detectors, then pass through the magnetic field and are deflected according to their charge, mass, and momentum. The third set tracks the particles during deflection and then the last two detect them after deflection. This set-up is particularly useful for studying di-muon events as the muons produced would be a muon-antimuon pair, which would deflect in opposite directions in the dipole magnet. This would leave a characteristic track signature that can be studied.

After the last tracking detector, particles pass through another absorber, which serves to further filter out muons from background, as well as filter out background muons. The muons produced in heavy quark resonances have considerably higher $\pt$ than those produced by background processes, so the job of the trigger system is to only trigger on muons with high enough $\pt$ to be interesting (this is defined per process) and the second absorber helps reduce the number of background muons incident on the trigger system. The trigger system is made of 2 stations of 2 Resistive Plate Chamber (RPC) detectors. Comparing the measurement of the same particle in the two stations, the $\pt$ can be determined. The decision to keep or reject an event takes about \SI{300}{\nano\second}.

In Run 1 and Run 2, the MCH performed all its own tracking and vertexing on the particles it studied. Particularly for vertexing, where the collision position is estimated, this was not optimal as most of the particles produced in the interactions didn't make it through to sensitive material. With the increased energy and luminosity of Run 3 a better system was needed to perform these tasks, so the MFT was added in front of the first absorber to take over the job.


\subsection{The Muon Forward Tracker}
The MFT is a brand new detector added to ALICE for Run 3 to assist the MCH with tracking and vertexing. It covers the range $-3.6<\eta<-2.45$ and was designed in conjunction with the ITS, using precisely the same CMOS pixel detectors. Due to the MFT being placed in front of the absorber, it detects a lot more particles than make it through to the MCH, allowing it to be much better at finding the primary vertex of collisions. \Cref{fig:MFT Schematic} shows the design of the MFT. The rest of this section is adapted from the MFT Technical Design Report~\cite{MFT_TDR}. 

\begin{figure}[h]
    \begin{center}
        \includegraphics[width=.8\textwidth]{Figs/MFT_schematic.jpg}
        \caption{Schematic view of the Muon Forward Tracker~\cite{MFT_Schematic}. The small cone on the left shows the IP. Note that the 5 disks each have a front and back plane of pixel detectors, totalling 10 $z$-positions for the MFT to ``see'' particles at.}
        \label{fig:MFT Schematic}
    \end{center}
\end{figure}

The MFT is made of two identical half-cones sandwiching the beam pipe from above and below, each with 5 half-disks positioned at different distances from the IP along the $z$-axis. Each half-disk has a front and back detection plane totalling 10 detection planes. As the disks get further from the IP their radius increases in order to cover the same $\eta$ range, aside from the second disk, which is identical to the first. The disks sit at $z$-positions -46.0, -49.3, -53.1, -68.7, and -76.8 \si{\centi\metre} respectively and each disk is \SI{1.4}{\centi\metre} thick, leading to detector planes at $\pm \SI{0.7}{\centi\metre}$ from each of those positions.

\begin{figure}[h]
    \begin{center}
        \includegraphics[width=.8\textwidth]{Figs/MFT_Disk4_mapping.png}
        \caption{The layout of pixel detector elements on the front and back plane of one of the half-disks of disk 4 in the MFT. Note that the front and back plane have layouts offset to each other by half the width of the pixel elements. \textbf{\textit{SOMETHING WEIRD ABOUT THIS PLOT. I THINK X AND Y ARE WRONG}} \cite[fig.~6.1]{MFT_TDR}}
        \label{fig:MFT_Disk4_mapping}
    \end{center}
\end{figure}

The half-disks are made from ladders of 2 to 5 rectangular pixel detector elements. \Cref{fig:MFT_Disk4_mapping} shows an example of the layout of the front and back planes of detector elements in both $x$-$y$ and $\eta$-$\varphi$. 




\subsection{The Online-Offline Analysis Framework}
With the increased interaction rate expected for Run 3, a new system for real-time processing, as well as offline analysis, needed to be constructed \cite{ALICE_Upgrade_LOI}. The Online-Offline (O2) framework was developed for this purpose. The ``Online'' portion of the framework is the real-time processing, where continuously captured data from ALICE is split into \SI{10}{\milli\second} chunks, called timeframes. These timeframes are then later processed, or ``reconstructed'' into Event Summary Data (ESD) and then Analysis Object Data (AOD) files---this is the ``Offline'' part---which contain only the data of interest. Different passes of reconstruction can be done for purposes and are often iterated upon if something was missed in a previous pass. Because of this, the same data can often look different between reconstruction passes. The AOD files can then be analysed---also Offline---through an ``Analysis Task'', written in C\OldTexttt{++} and ROOT. 

We distinguish between data taken in Run 3 and data taken in Run 1 and 2 by calling Run 3 AOD's ``AO2D'' files, and Run 1 and 2 just ``AOD'' files. The data is stored on the \OldTexttt{alimonitor} system, requiring a certificate to access, which is obtained by joining the ALICE collaboration. Access to all data and most analysis tools used in this report is restricted behind this wall. 

\begin{figure}[h]
    \begin{center}
        \includegraphics[width=.6\textwidth]{Figs/O2_flow.png}
        \caption{Functional flow of the O2 framework~\cite{O2_Upgrade_TDR}. Detectors output their signal continuously in the new configuration and this signal gets split into chunks called timeframes. These timeframes get processed a number of times, reducing the volume of data each time by only extracting the quantities that will be used for analysis. Many choices have to be made at each step to ensure that useful data makes it out the other side and so the details of this process are always in flux.}
        \label{fig:O2_flow}
    \end{center}
\end{figure}

The focus of the upgraded analysis framework was to reduce disk space usage when processing and analysing, as well as making sure all analysis takes advantage of all processing power available to it at all times. \Cref{fig:O2_flow} shows the general flow of data in the O2 processing pipeline from particle hits in detectors until usable data in the form of AOD files. 



\section{Analysing Using O2}\label{sec:AnalysingWithO2}
Writing an analysis task in O2 is structured quite strictly, due to it needing to be run by O2. We will outline the steps needed to go from reconstructed data to histograms of, in our case, kinematic variables. It must be noted that learning how to do this, and how to deal with the idiosyncrasies of the O2 software, were what took up the majority of the time spent on this project. It is written to do one thing very very well, but unfortunately that comes with the side-effect of it being extremely picky about the conditions in which the software will actually work. 

\subsection{AOD Structure}\label{sec:AODStructure}
The data that we use in our analysis comes in the form of Analysis Object Data. These are in the form of ROOT files, which are based on a tree structure. In the tree are a number of tables corresponding to reconstructed tracks and collisions, among a few others. We call these entries or rows. For each of these tables, there are a number of reconstructed variables associated with each entry, such as their $\pt$, $\eta$, or $\varphi$, which we can access. There are 4 types of variables:
\begin{itemize}
    \item Static variables are saved to disk during the reconstruction process and are available at any time. $\varphi$ is a static variable.
    \item Dynamic variables are calculated on demand, using static variables as input. $\eta$ is a dynamic variable, calculated from $\theta$. 
    \item Index variables point from one table to another, such as from a track to its associated collision.
    \item Expression variables no clue
\end{itemize}

All variables associated with a track, for example, could be included in a single table. However, since we often only need a few of the variables, the tables are split up into sections that contain variables often used together. If needed, these tables can be joined together when doing analysis, using \mintinline{c++}{o2::soa::Join<Table1, Table2>}. Importantly, only tables which correspond row-to-row and have the same number of rows can be joined in this way.

\subsection{Analysis Task Structure}\label{sec:TaskStructure}
Analysis tasks are written in C\texttt{++} and need to be structured in a specific way for O2 to use them properly. Each task is written as a \Cppinline{struct} object which is then called at the end of the task. Below is an outline of what is needed for a task.

\begin{minted}{c++}
#include "Framework/runDataProcessing.h"
#include "Framework/AnalysisTask.h"

struct MyTask {
    // Define things here, such as histogram registries, filters for data, or new tables

    void init{o2::framework::InitContext&} {
        // Here we initialise histograms and other things used in the analysis
    };

    void process(aod::Collisions const& collisions, aod::Tracks const& tracks) {
        // Here we can do any processing that we need, calculating things etc, and then fill the histograms we defined earlier
    };
};

// This is what O2 looks at to run the task
WorkflowSpec defineDataProcessing(ConfigContext const& cfgc)
{
  return WorkflowSpec{
    adaptAnalysisTask<MyTask>(cfgc),
  };
}

\end{minted}

With the task written, it then needs to be compiled and added to O2 so that it can be run. O2Physics has a number of analysis tasks written by people at ALICE which get compiled automatically. These are sorted into physics working groups such as Heavy Flavour (PWGHF) and Electromagnetism (PWGEM). If we want to add our own task to O2Physics, we need to create our own folder with the same structure as the working groups. Below shows the structure of the file system.

\dirtree{%
  .1 alice.
  .2 alidist.
  .2 O2.
  .2 O2Physics.
  .3 Functional Things.
  .3 ....
  .3 myTasks.
  .4 CMakeLists.txt.
  .4 Tasks.
  .5 CMakeLists.txt.
  .5 analysisTask.cxx.
  .5 ....
  .3 Other Working Groups.
  .3 ....
  .3 CMakeLists.txt.
}

Here \texttt{alidist} is the git repository that handles the versioning of O2 and O2Physics. O2 is what handles the backend of the analysis framework, compiling the tasks written in O2Physics. The \texttt{CMakeLists.txt} files are needed at every level of the O2Physics structure to tell O2 what to compile and which commands to use to refer to things. 

\textbf{\textit{PUT AN EXAMPLE CMAKELISTS HERE}}



\subsection{Compiling O2 \& Running a Task}\label{sec:CompileRun}
O2 is built on your system using \texttt{aliBuild}~\cite{aliBuild_install}. It prefers to be built on UNIX systems and requires at least 8 GB of RAM, preferably more. We used both O2 and O2Physics as we wanted to create and run analysis tasks. 

Once O2 is built (those four words are doing some \textit{very} heavy lifting) we can enter the O2Physics environment with \texttt{alienv enter O2Physics} and this will place us in a new terminal shell. The magic of O2 is that it compiles all the analysis tasks in O2Physics, as well as tools for simulation and the like, such that everything can be done by running commands in that shell. Before running our own tasks, however, we need to tell O2 to build our tasks into O2Physics. To do this we use \texttt{ninja}.




For our purposes, the only commands we need to know are how to run a task and the options that come along with that. 

All analysis tasks in O2Physics get assigned a unique command that can be used to run that task. They all begin with \texttt{o2-analysis...} followed by the name assigned to it in the relevant \texttt{CMakeLists.txt} file. Most analysis tasks are run on \texttt{AOD.root} or \texttt{AO2D.root} files and to tell the task which file to use, we use the flag \texttt{--aod-file AO2D.root}. We could also supply a list of files in a text file and use \texttt{--aod-file @AO2D\_list.txt} where \texttt{AO2D\_list.txt} contains the path to the files we want to run on. 

Lastly on the topic of running tasks is piping the output of one task into another. Often we want to run multiple tasks in succession on the same data, feeding the output of one into another. To do this, we simply use the pipe symbol \texttt{|} between the tasks: \texttt{o2-analysis-trackselection | o2-analysis-ud-mytask --aod-file AO2D.root}. Here the ordering of the tasks doesn't matter as the input and output format of a task is known before it runs, so O2 does some quick thinking to arrange the workflow such that the tasks get fed the correct format of data.


\section{Our Analysis}\label{sec:OurAnalysis}
The intention of this report is to investigate preliminary data coming from the ITS and MFT in Run 3. The order of the discussion in this section will follow the order that we tackled things. This is done to show both the progression of our knowledge as well as to try to clarify some of the explanations in previous sections as the only way to truly understand some things is through example. 

\subsection{The Data}
The data analysed in this report was taken in October 2021, where protons were collided at a centre of mass energy of \SI{900}{\giga\electronvolt}. This is not an energy that we expect to use for physics research but it allows us to look at how the detectors are performing with more lightweight data, simply because there will be fewer particles created in the collisions and thus less data to work with. We are using runs 505548 and 505645. In this case, a run specifies a period of data taking for which all global settings remain the same. 



\subsection{Initial MFT Analysis}
In the AOD data model there is a table called \texttt{MFTTracks} which contains the tracks detected in the MFT. 

When we began this analysis, only two reconstruction passes had been run on the data and while the MFT was switched on for the runs, 

\subsection{Comparing pass3 to pass4}


\subsection{Initial ITS Analysis}


\subsection{Using pass4 for hasITS}

\section{Discussion and Recommendations}


There were many times during our analysis where we came up against a limitation in the analysis software that restricted our ability to see the details we were after. This really comes down to the fact that the types of analysis that O2 was (and still is being) built for are not the type that we were looking to do. Given more time to attain a greater understanding of the inner workings of particularly the steps from CTFs to AODs, we might have been able to delve into the belly of the beast and find the data ourselves. Since we were limited by our time this year, the best we can do is recommend some modifications to the AOD data model that would lend themselves to these analyses. 

Firstly we recommend introducing more detail about the tracks in the MFT, such as the position of hits in each disk, or even a simple ``Cluster Map''-style column in the \texttt{aod::MFTTracks} table to be able to see which disks or detector planes contributed to the track. At the present moment there is no way to view the vertex position data as determined by the ITS or the MFT alone. Adding this capability would allow for a much deeper investigation into the improvement of the new ITS and MFT in that sector than we were able to do.

Outside the realm of modifying O2 and the data model, there is more that could be done to properly verify if the distributions we see for, say $\eta$ and $\varphi$, are correct. We have attempted to explain them from a physical motivation but comparison to simulated data would prove the most robust way to check if there are no large systematic errors. In the interest of determining how the MFT has improved the tracking capabilities in the forward region, the old techniques used in Runs 1 and 2 using the MCH could be implemented on new data and compared to the new technique on the same data. This would again require investigation at a much more involved level than was possible in this report but could lead to some interesting results. Once again simulation could aid in this endeavour. 

Lastly, we were left with many questions about the details of the track reconstruction, read-out architecture, and columns in the data model such as the $z$ variable in \texttt{aod::MFTTracks}. This effectively comes down to the documentation surrounding these detectors and analysis framework, so a concerted effort to improve that documentation would be greatly appreciated by people new to the ALICE group and to these detectors. 

\section{Conclusions}



\newpage
\printbibliography

\newpage
\section*{Appendix}
\appendix
\section{Downloading data from the GRID}\label{apx:Downloading_Data}
\begin{minted}{bash}
    # The directory in alimonitor where you want to get the data from. Should contain a load of numbered directories
    # For example, /alice/data/2021/OCT/505548/AOD. Note that it's AOD not AO2D
    sourceMotherDir=/alice/data/path/to/AOD
    nFiles=$1

    # This is the directory on your local machine where you want to store your data
    targetMotherDir=/path/to/alice/data/${sourceMotherDir}
    mkdir -p ${targetMotherDir}

    for ((i=1; i<=nFiles; i++))
    do
        if (( i < 10 )); then
            pref=00
        fi
        if (( i > 9 )); then
            pref=0
        fi
        iDir=${pref}${i}
        iSourceDir=${sourceMotherDir}/${iDir}
        iTargetDir=${targetMotherDir}/${iDir}
        mkdir -p ${iTargetDir}

        echo "copying from ${iSourceDir} to ${iTargetDir}" 
        
        alien_cp -retry 5 ${iSourceDir}/AO2D.root file://${iTargetDir}/AO2D.root
    done
\end{minted}

\newpage
\section{Additional Plots}\label{apx:Additional_Plots}

\begin{figure}[h]%
    \centering
    \begin{subfigure}[t]{.49\linewidth}
        \centering
        \includegraphics[width=\linewidth]{Plots/pass4_MFT/MFTeta_pass4.pdf}
        \caption{}
        \label{}
    \end{subfigure}
    \hfill
    \begin{subfigure}[t]{.49\linewidth}
        \centering
        \includegraphics[width=\linewidth]{Plots/pass4_MFT/phi_pass4.pdf}
        \caption{}
        \label{}
    \end{subfigure}
    \begin{subfigure}[t]{.49\linewidth}
        \centering
        \includegraphics[width=\linewidth]{Plots/pass4_MFT/nClusters_pass4.pdf}
        \caption{}
        \label{}
    \end{subfigure}
    \hfill
    \begin{subfigure}[t]{.49\linewidth}
        \centering
        \includegraphics[width=\linewidth]{Plots/pass4_MFT/Z_MFT_pass4.pdf}
        \caption{}
        \label{}
    \end{subfigure}
\caption[Histograms of $\eta$, $\varphi$, \OldTexttt{nClusters}, and $z$ for tracks from pass 4 in the MFT]{Histograms of kinematic variables for tracks in the MFT from reconstruction pass 4. Presented here for comparison to \cref{fig:MFT_1D_pass3}. Note that other than $\eta$ there is no significant difference between these and those from pass 3. }
\label{fig:MFT_1D_pass4}
\end{figure}

\begin{figure}[h]%
    \centering
    \begin{subfigure}[t]{.49\linewidth}
        \centering
        \includegraphics[width=\linewidth]{Plots/pass4_MFT/x_y_1_pass4.pdf}
        \caption{}
        \label{}
    \end{subfigure}
    \hfill
    \begin{subfigure}[t]{.49\linewidth}
        \centering
        \includegraphics[width=\linewidth]{Plots/pass4_MFT/x_y_2_pass4.pdf}
        \caption{}
        \label{}
    \end{subfigure}
    \begin{subfigure}[t]{.49\linewidth}
        \centering
        \includegraphics[width=\linewidth]{Plots/pass4_MFT/x_y_3_pass4.pdf}
        \caption{}
        \label{}
    \end{subfigure}
    \hfill
    \begin{subfigure}[t]{.49\linewidth}
        \centering
        \includegraphics[width=\linewidth]{Plots/pass4_MFT/x_y_4_pass4.pdf}
        \caption{}
        \label{}
    \end{subfigure}
\caption[$x$-$y$ histograms of tracks at different planes in the MFT]{$x$-$y$ histograms for MFT tracks from reconstruction pass 4. Once again there is not much difference compared to pass 3 other than a general trend towards more uniform data. }
\label{fig:MFT_2D_pass4}
\end{figure}

\end{document}



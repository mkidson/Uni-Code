\documentclass[12pt]{article}
\usepackage[margin=1.2in]{geometry}
\usepackage[all]{nowidow}
\usepackage[hyperfigures=true, hidelinks, pdfhighlight=/N]{hyperref}
\usepackage[separate-uncertainty=true]{siunitx}
\usepackage{graphicx,amsmath,physics,tabto,float,amssymb,pgfplots,verbatim,tcolorbox}
\usepackage{listings,xcolor,subfig,keyval2e,caption,import}
\numberwithin{equation}{section}
\numberwithin{figure}{section}
\definecolor{stringcolor}{HTML}{C792EA}
\definecolor{codeblue}{HTML}{2162DB}
\definecolor{commentcolor}{HTML}{4A6E46}
\lstdefinestyle{appendix}{
    basicstyle=\ttfamily\footnotesize,commentstyle=\color{commentcolor},keywordstyle=\color{codeblue},
    stringstyle=\color{stringcolor},showstringspaces=false,numbers=left,upquote=true,captionpos=t,
    abovecaptionskip=12pt,belowcaptionskip=12pt,language=Python,breaklines=true,frame=single}
\lstdefinestyle{inline}{
    basicstyle=\ttfamily\footnotesize,commentstyle=\color{commentcolor},keywordstyle=\color{codeblue},
    stringstyle=\color{stringcolor},showstringspaces=false,numbers=left,upquote=true,frame=tb,
    captionpos=b,language=Python}
\renewcommand{\lstlistingname}{Appendix}
\pgfplotsset{compat=1.17}

\title{IA Assignment 3}
\author{KDSMIL001 \; 2IA MAM2000W}
\date{\textbf{7 September 2020}}

\begin{document}
    \maketitle
    \begin{enumerate}
        \item \begin{enumerate}
            \item First of all, we use the basic division theorem:
            \begin{align*}
                47&=(1)27+20\\
                27&=(1)20+7\\
                20&=(2)7+6\\
                7&=(1)6+1
            \end{align*}
            So we know that $\gcd(47,27)=1$, i.e. they are coprime, so there exists a number $b$ 
            such that $1=27b+47k$, for some integer k, which is the inverse of $27\pmod{47}$, or 
            more generally $[27]^{-1}$. We can find this $b$ using the extended division algorithm:
            \begin{align*}
                1&=7-6\\
                &=7-(20-(2)7)\\
                &=(3)7-20\\
                &=(3)(27-20)-20\\
                &=(3)27-(4)20\\
                &=(3)27-(4)(47-27)\\
                &=(7)27-(4)47
            \end{align*}
            and so we have $27^{-1}=7$ (mod 47).
            
            \item Given the congruence $[135][x]=[15]\pmod{235}$ we must first check if 135 and 235 
            are coprime. To do this we use the division algorithm:
            \begin{align*}
                235&=135+100\\
                135&=100+35\\
                100&=(2)35+30\\
                35&=30+5\\
                30&=(6)5
            \end{align*}
            So they are not coprime, but we can see that our congruence can be divided by 5, so we have 
            \begin{equation*}
                [27][x]=[3] \pmod{47}
            \end{equation*}
            To solve this we need the inverse of $27\pmod{47}$, but we already know this from (a), so we have 
            \begin{align*}
                [x]&=[3][27]^{-1}\\
                &=[3][7]\\
                &=[3\cdot7]\\
                &=[21]
            \end{align*}
            And we're done.
            
            \item As we saw above in (b), 135 and 235 are not coprime, but this time we have the 
            congruence 
            \begin{equation*}
                [135][x]=[14] \pmod{235}
            \end{equation*}
            and $5\nmid14$, so there is no solution for $x$.
        \end{enumerate}
    \end{enumerate}

\end{document}
\documentclass[12pt]{article}
\usepackage[margin=1in]{geometry}
\usepackage[all]{nowidow}
\usepackage[hyperfigures=true, hidelinks, pdfhighlight=/N]{hyperref}
\usepackage{graphicx,amsmath,physics,tabto,float,amssymb,pgfplots,verbatim,tcolorbox}
\usepackage{listings,xcolor,siunitx,subfig,keyval2e,caption,import}
\numberwithin{equation}{section}
\numberwithin{figure}{section}
\definecolor{stringcolor}{HTML}{C792EA}
\definecolor{codeblue}{HTML}{2162DB}
\definecolor{commentcolor}{HTML}{4A6E46}
\lstdefinestyle{appendix}{
    basicstyle=\ttfamily\footnotesize,commentstyle=\color{commentcolor},keywordstyle=\color{codeblue},
    stringstyle=\color{stringcolor},showstringspaces=false,numbers=left,upquote=true,captionpos=t,
    abovecaptionskip=12pt,belowcaptionskip=12pt,language=Python,breaklines=true,frame=single}
\lstdefinestyle{inline}{
    basicstyle=\ttfamily\footnotesize,commentstyle=\color{commentcolor},keywordstyle=\color{codeblue},
    stringstyle=\color{stringcolor},showstringspaces=false,numbers=left,upquote=true,frame=tb,
    captionpos=b,language=Python}
\renewcommand{\lstlistingname}{Appendix}
\pgfplotsset{compat=1.17}

\title{Tutorial 1}
\date{\textbf{25 August 2020}}
\author{KDSMIL001 \; MAM2046W 2BP}

\begin{document}
    \maketitle
    \begin{enumerate}
        \item Given the differential equation $y''+\lambda y=0$ we can guess the solution $y=e^{rx}$, 
        which will give us the characteristic equation
        \begin{equation*}
            r^2+\lambda=0
        \end{equation*}
        This has different solutions depending on the value of $\lambda$:
        \begin{itemize}
            \item $\lambda>0 \implies r=\pm \sqrt{\lambda}i,\; i^2=-1$\newline
            This gives us a general solution for the DE of
            \begin{align*}
                y(x)&=Ae^{i\sqrt\lambda x}+Be^{-i\sqrt\lambda x}\\
                &= C_1\cos(\sqrt\lambda x)+C_2\sin(\sqrt\lambda x)
            \end{align*}
            Now we apply the boundary conditions. First, $y(-3)=y(3)$:
            \begin{align*}
                C_1\cos(-\sqrt{\lambda} 3)+C_2\sin(-\sqrt{\lambda} 3)&=C_1\cos(\sqrt{\lambda} 3)+C_2\sin(\sqrt{\lambda} 3)\\
                C_1\cos(\sqrt{\lambda} 3)-C_2\sin(\sqrt{\lambda} 3)&=C_1\cos(\sqrt{\lambda} 3)+C_2\sin(\sqrt{\lambda} 3)\\
                2C_2\sin(\sqrt{\lambda} 3)&=0\\
                C_2\sin(\sqrt{\lambda} 3)&=0
            \end{align*}
            We can't say for sure that $C_2=0$ or that $\sin(\sqrt{\lambda}3)=0$, so we continue.\newline
            Now $y'(-3)=y'(3)$. Firstly we see that $y'(x)=\sqrt{\lambda}(-C_1\sin(\sqrt{\lambda} 3)+C_2\cos(\sqrt{\lambda} 3))$
            \begin{align*}
                \sqrt{\lambda}(-C_1\sin(-\sqrt{\lambda} 3)+C_2\cos(-\sqrt{\lambda} 3))&=\sqrt{\lambda}(-C_1\sin(\sqrt{\lambda} 3)+C_2\cos(\sqrt{\lambda} 3))\\
                -C_1\sin(-\sqrt{\lambda} 3)+C_2\cos(-\sqrt{\lambda} 3)&=-C_1\sin(\sqrt{\lambda} 3)+C_2\cos(\sqrt{\lambda} 3)\\
                C_1\sin(\sqrt{\lambda} 3)+C_2\cos(\sqrt{\lambda} 3)&=-C_1\sin(\sqrt{\lambda} 3)+C_2\cos(\sqrt{\lambda} 3)\\
                2C_1\sin(\sqrt{\lambda} 3)&=0\\
                C_1\sin(\sqrt{\lambda} 3)&=0
            \end{align*}
            From these two results we can say that either $C_1=C_2=0$, which is the trivial solution, or 
            \begin{align*}
                \sqrt{\lambda}3&=n\pi,\; n\in\mathbb{N}\\
                \implies \lambda_n&=\frac{n^2\pi^2}{9}
            \end{align*}
            which are the eigenvalues for this solution, the corresponding eigenfunctions being 
            \begin{equation*}
                y_n(x)=C_1\cos(\frac{n\pi}{3}x)+C_2\sin(\frac{n\pi}{3}x),\; n\in\mathbb{N}
            \end{equation*}

            \item $\lambda=0\implies r=0$ is our only, and therefore repeated, root. This gives us
            \begin{align*}
                y(x)&=Ae^{rx}+Bxe^{rx}\\
                &=A+Bx
            \end{align*}
            Applying the first initial condition $y(-3)=y(3)$ we get 
            \begin{align*}
                A-3B&=A+3B\\
                6B&=0\\
                B&=0\\
                \implies y(x)&=A,\; A\in\mathbb{R}
            \end{align*}
            We can choose any $A$, so we choose 1. Thus our eigenvalue is 0 and our eigenfunction is $y(x)=1$.

            \item $\lambda<0 \implies r=\pm\sqrt{-\lambda}$ where both roots are real. Substituting this into 
            our original guess we get
            \begin{align*}
                y(x)&=Ae^{\sqrt{-\lambda}x}+Be^{-\sqrt{-\lambda}x}\\
                &=C_1\cosh(\sqrt{-\lambda}x)+C_2\sinh(\sqrt{-\lambda}x)
            \end{align*}
            Now we apply the BCs, starting with $y(-3)=y(3)$:
            \begin{align*}
                C_1\cosh(-\sqrt{-\lambda}3)+C_2\sinh(-\sqrt{-\lambda}3)&=C_1\cosh(\sqrt{-\lambda}3)+C_2\sinh(\sqrt{-\lambda}3)\\
                C_1\cosh(\sqrt{-\lambda}3)-C_2\sinh(\sqrt{-\lambda}3)&=C_1\cosh(\sqrt{-\lambda}3)+C_2\sinh(\sqrt{-\lambda}3)\\
                2C_2\sinh(\sqrt{-\lambda}3)&=0\\
                C_2\sinh(\sqrt{-\lambda}3)&=0
            \end{align*}
            $\sinh(x)=0$ in one case only: $x=0$. That means either $\sqrt{-\lambda}=0$, which can't be as $\lambda>0$, or 
            $C_2=0$, which must be the case. \newline
            Now we look at the second BC. Firstly we know that 
            \begin{align*}
                y'(x)&=\sqrt{-\lambda}(C_1\sinh(\sqrt{-\lambda}x)+C_2\cosh(\sqrt{-\lambda}x))\\
                &=\sqrt{-\lambda}C_1\sinh(\sqrt{-\lambda}x)
            \end{align*}
            Applying our BC of $y'(-3)=y'(3)$ gives us
            \begin{align*}
                \sqrt{-\lambda}C_1\sinh(-\sqrt{-\lambda}3)&=\sqrt{-\lambda}C_1\sinh(\sqrt{-\lambda}3)\\
                -C_1\sinh(\sqrt{\lambda}3)&=C_1\sinh(\sqrt{\lambda}3)\\
                2C_1\sinh(\sqrt{\lambda}3)&=0\\
                C_1\sinh(\sqrt{\lambda}3)&=0
            \end{align*}
            Again we can see that $C_1=0$ as $\sinh$ can't be 0 for the same reasons as before. This means 
            that this case has no eigenvalues and no eigenfunctions.
        \end{itemize}

        \item To begin, we guess a solution $y(x)=x^r$ for $x>r$. We differentiate this to get 
        $y'(x)=rx^{r-1},\;y''(x)=r(r-1)x^{r-2}$. Plugging these back in we get 
        \begin{align*}
            0&=x^2x^{r-2}r(r-1)+\lambda(rx^{r-1}x-x^r)\\
            &=rx^r(r-1)+\lambda x^r(r-1)\\
            &=x^r(r-1)(r+\lambda)
        \end{align*}
        We know that $x^r\neq0$ so our two roots are $r=1,\; r=-\lambda$. Using these values gives us two 
        linearly independent solutions: $y(x)=x$ and $y(x)=x^{-\lambda}$. These can be combined to find a 
        general solution:
        \begin{align*}
            y(x)&=C_1x+C_2x^{-\lambda}\\
            \implies y'(x)&=C_1-C_2\lambda x^{-\lambda-1}
        \end{align*}
        Applying our Mixed BCs $y(1)=0,\; y'(e)=0$ we get 
        \begin{align*}
            y(1)=0&=C_1+C_2\\
            \implies C_1&=-C_2\\
            y'(e)=0&=C_1-C_2\lambda e^{-\lambda-1}\\
            \implies C_1&=C_2\lambda e^{-\lambda-1}\\
            \implies -C_2&=C_2\lambda e^{-\lambda-1}\\
            \implies -1&=\lambda e^{-\lambda-1}
        \end{align*}
        The only solution to this is $\lambda=-1$, and thus we have found our one and only real eigenvalue. 
        Its corresponding eigenfunction is found by simply substituting this into the general equation to find 
        \begin{equation*}
            y_1(x)=C_1x+C_2x^1=C_3x,\; \lambda_1=-1
        \end{equation*}
        
        \item Given the PDE 
        \begin{equation*}
            \frac{\partial C}{\partial t}=\frac{\partial^2 C}{\partial x^2}-hC
        \end{equation*}
        where $h>0$ is a constant and $C(x,t)$ describes a concentration of chemicals, with boundary conditions 
        $C(0,t)=0, C(L,t)=0, t>0$ and initial condition $C(x,0)=x$ for $0<x\leq \frac{L}{2}$ and $C(x,0)=0$ for 
        $\frac{L}{2}<x<L$:
        \begin{enumerate}
            \item We solve it using separation of variables. This starts by splitting the function $C(x,t)$ into 
            two functions dependent on $x$ and $t$ separately: $C(x,t)=\phi(x)G(t)$. Because these functions are 
            independent of each other, we can write the original PDE as
            \begin{align*}
                \dot{G}\phi&=\phi'' G-h\phi G\\
                \implies \frac{\dot{G}}{G}&=\frac{\phi''-h\phi}{\phi}=-\lambda
            \end{align*}
            where $\lambda$ is arbitrary. We use this $\lambda$ to solve each ODE separately, but first we must 
            examine our BCs as they are affected by this separation of variables:
            \begin{align*}
                C(0,t)=0&\implies \phi(0)G(t)=0\implies\phi(0)=0\\
                C(L,t)=0&\implies \phi(L)G(t)=0\implies\phi(L)=0
            \end{align*}
            Now we solve $\dot{G}=-\lambda G$, which is easily seen to be $G(t)=Ae^{-\lambda t}$. We know that 
            this function will govern $C$ and it makes sense that the concentration of chemicals will decrease as 
            time goes on, not increase, so we can assume that $\lambda>0$.\newline
            Next we move to the more complicated $\phi'' -\phi(h-\lambda)=0$. If we guess the solution $y=e^{rx}$ 
            then its characteristic equation is $r^2-(h-\lambda)=0$. We know that $\lambda>0$ so we examine 
            three cases: 
            \begin{itemize}
                \item $\lambda>h \implies r=\pm \sqrt{\lambda-h}i$. Our guess leads us to a general solution
                \begin{align*}
                    \phi(x)&=Ae^{i\sqrt{\lambda-h}x}+Be^{-i\sqrt{\lambda-h}x}\\
                    &=C_1\cos(\sqrt{\lambda-h}x)+C_2\sin(\sqrt{\lambda-h}x)
                \end{align*}
                Applying our BCs we get 
                \begin{align*}
                    \phi(0)=0&=C_1\cos(0)+C_2\sin(0)\\
                    0&=C_1\\
                    \phi(L)=0&=C_2\sin(\sqrt{\lambda-h}L)\\
                    \implies \sqrt{\lambda-h}L&=n\pi,\; n\in\mathbb{N} \\
                    \implies \lambda_n&=\frac{n^2\pi^2}{L^2}+h,\; n\in\mathbb{N}
                \end{align*}
                We can make the assumption that $C_2\neq 0$ because having both constants 0 is the trivial solution, 
                so the $\sin$ term must be 0. With this infinite family of eigenvalues we can find their corresponding 
                eigenfunctions:
                \begin{equation*}
                    \phi_n(x)=C_2\sin(\frac{n\pi}{L}x)
                \end{equation*}

                \item $\lambda=h \implies r^2-0=0 \implies r=0$ is our repeated root. Our general solution becomes 
                \begin{align*}
                    \phi(x)&=Ae^{rx}+Bxe^{rx}\\
                    &=A+Bx
                \end{align*}
                Applying our BCs we get 
                \begin{align*}
                    \phi(0)=0&=A\\
                    \phi(L)=0&=BL
                \end{align*}
                We know by the construction of the problem that $L\neq0$ so it must be that $B=0$. Both our constants 
                are 0 so there are no eigenvalues or eigenfunctions.

                \item $0<\lambda<h \implies r=\pm\sqrt{h-\lambda}$. Our general solution becomes 
                \begin{align*}
                    \phi(x)&=Ae^{\sqrt{h-\lambda}x}+Be^{-\sqrt{h-\lambda}x}\\
                    &=C_1\cosh(\sqrt{h-\lambda}x)+C_2\sinh(\sqrt{h-\lambda}x)
                \end{align*}
                Applying the BCs we find 
                \begin{align*}
                    \phi(0)=0&=C_1\cosh(0)+C_2\sinh(0)\\
                    &=C_1\\
                    \phi(L)=0&=C_2\sinh(\sqrt{h-\lambda}L)\\
                    \implies C_2&=0
                \end{align*}
                as $\sinh(x)=0$ at $x=0$ only, and neither $\sqrt{h-\lambda}$ or $L$ are 0, so we have the trivial 
                solution: No eigenvalues or eigenfunctions.
            \end{itemize}
            Finally we can combine our two functions back into $C(x,t)$:
            \begin{equation*}
                C(x,t)=\phi(x)G(t)=C\sin(\frac{n\pi}{L}x)e^{-\lambda_n t}
            \end{equation*}
            where $C=C_1A,\; \lambda_n=\frac{n^2\pi^2}{L^2}+h,\; n\in\mathbb{N}$. This tells us there is an infinite 
            family of solutions to this PDE, meaning we can construct another solution by taking a linear combination 
            of them all:
            \begin{equation*}
                C(x,t)=\sum_{n=1}^{\infty}C_n \sin(\frac{n\pi}{L}x) e^{-\left(\frac{n^2\pi^2}{L^2}+h\right)t}
            \end{equation*}
            And now it's time to apply our initial conditions, which are 
            \begin{align*}
                C(x,0)&=x \;\text{for}\; 0<x\leq \frac{L}{2}\\
                C(x,0)&=0 \;\text{for}\; \frac{L}{2}<x<L
            \end{align*}
            Applying these to our linear combination solution we have 
            \begin{align*}
                C(x,0)=x&=\sum_{n=1}^{\infty} C_n\sin(\frac{n\pi}{L}x) e^0,\; 0<x\leq \frac{L}{2}\\
                C(x,0)=0&=\sum_{n=1}^{\infty} C_n\sin(\frac{n\pi}{L}x) e^0,\; \frac{L}{2}<x<L
            \end{align*}
            For $\frac{L}{2}<x<L$, either all $C_n=0$, or $\frac{n\pi}{L}x=0 \implies x=L$ which isn't possible, 
            so $C_n=0$ for all $n$ in that case. Going back to the first case, eigenfunctions form a complete 
            orthogonal set, meaning we can find the value of $C_n$ in the following way:
            \begin{align*}
                C_n&=\frac{4}{L}\langle x | \sin(\frac{n\pi x}{L})\rangle\\
                &= \frac{4}{L} \int_0^{\frac{L}{2}} x\sin(\frac{n\pi x}{L})dx\\
                &= \frac{4}{L} \frac{L \left(L\sin(\frac{\pi n}{2})-\frac{1}{2}\pi Ln\cos(\frac{\pi  n}{2})\right)}{\pi^2n^2}\\
                &= \frac{4\left(L\sin(\frac{\pi n}{2})-\frac{1}{2}\pi Ln\cos(\frac{\pi  n}{2})\right)}{\pi^2n^2}
            \end{align*}
            Finally we have a complete solution:
            \begin{equation*}
                C(x,t)=\sum_{n=1}^{\infty}\frac{4\left(L\sin(\frac{\pi n}{2})-\frac{1}{2}\pi Ln\cos(\frac{\pi  n}{2})\right)}{\pi^2n^2}\sin(\frac{n\pi}{L}x)e^{-\left(\frac{n^2\pi^2}{L^2}+h\right)t}
            \end{equation*}

            \item From the equation above, it's clear to see that 
            \begin{equation*}
                t\rightarrow\infty \implies C(x,t)\rightarrow\sum_{n=1}^{\infty}\frac{4\left(L\sin(\frac{\pi n}{2})-\frac{1}{2}\pi Ln\cos(\frac{\pi  n}{2})\right)}{\pi^2n^2}\sin(\frac{n\pi}{L}x)
            \end{equation*}
        \end{enumerate}

        \item 
        \begin{enumerate}
            \item This problem can be imagined as a rod that begins with a distribution of heat along it relative 
            whose one end is perfectly insulated and begins at temperature 0, and whose other end is perfectly 
            insulated and begins at some temperature.

            \item Firstly we expand the given PDE into a more useful form:
            \begin{align*}
                \mathcal{H}u&=0\\
                \implies \frac{\partial u}{\partial t}&=k\frac{\partial^2 u}{\partial x^2}\\
                \implies \dot{G}\phi &= kG\phi''
            \end{align*}
            where we have split $u(x,t)$ into two functions dependent on only $x$ and $t$ respectively, $G(t),\;\phi(x)$.
            Separating these "variables" we can recover 2 completely separate differential equations, so long 
            as the both equal the same constant, which we'll call $-\lambda$:
            \begin{align*}
                \dot{G}&=-kG\lambda\\
                \phi''&=-\phi\lambda
            \end{align*}
            Looking at $G(t)$ first, it's easy to see that the solution to this DE is $G(t)=Ae^{-k\lambda t}$. 
            If we think about this a bit more we can actually see that this equation will govern $u(x,t)$ 
            somewhat, meaning we can deduce some things about $\lambda$. Most importantly we can see that 
            $\lambda>0$ as we would expect that the heat in a bar of some kind, without any external sources, 
            will not increase with time and especially not at an exponential rate. \newline
            Next we'll look at $\phi(x)$. Luckily we only need to consider the case where $\lambda>0$, which 
            makes things easier. In fact we have already studied this exact form of equation in question 1, so 
            borrowing from that answer, we know that 
            \begin{align*}
                \phi(x)&=C_1\cos(\sqrt\lambda x)+C_2\sin(\sqrt\lambda x)\\
                \phi'(x)&=\sqrt{\lambda}(-C_1\sin(\sqrt{\lambda}x)+C_2\cos(\sqrt{\lambda}x))
            \end{align*}
            Now we consider the BCs, which are affected by our splitting of $u$ in the following way:
            \begin{align*}
                u_x(0,t)=0&=G'\phi+\phi' G=\phi'(0)G(t)=\phi'(0)\\
                \implies 0&=\sqrt{\lambda}(-C_1\sin(\sqrt{\lambda}0)+C_2\cos(\sqrt{\lambda}0))\\
                &=C_2\\
                u(L,t)=0&=G(t)\phi(L)=\phi(L)\\
                \implies 0&=C_1\cos(\sqrt\lambda L)\\
                \implies \sqrt\lambda L&=n\pi \\
                \implies \lambda_n =\frac{n^2\pi^2}{L^2},\; n\in\mathbb{N}
            \end{align*}
            We can assume that $C_1\neq0$ as that would just be the trivial solution. Thus we have found our 
            infinite family of eigenvalues, and their corresponding eigenfunctions are 
            \begin{equation*}
                \phi_n(x)=C_n\cos(\frac{n\pi}{L} x),\; n\in\mathbb{N}
            \end{equation*}
            Now we can reconstruct $u$ as 
            \begin{equation*}
                u(x,t)=G(t)\phi(x)=C_ne^{-k\frac{n^2\pi^2}{L^2} t}\cos(\frac{n\pi}{L} x),\; n\in\mathbb{N}
            \end{equation*}
            but because this is an infinite family of linearly independent solutions to $u$, a linear combination 
            of them all is also a solution to $u$, thus we have 
            \begin{equation*}
                u(x,t)=\sum_{n=1}^\infty C_ne^{-k\frac{n^2\pi^2}{L^2} t}\cos(\frac{n\pi}{L} x)
            \end{equation*}
            Now we can apply our initial condition $u(x,0)=x,\; 0<x<L$, giving us 
            \begin{equation*}
                u(x,0)=x=\sum_{n=1}^\infty C_ne^{0}\cos(\frac{n\pi}{L} x),\; 0<x<L
            \end{equation*}
            And now we can finally find a value for $C_n$ using the formula
            \begin{align*}
                C_n&=\frac{\langle x|\cos(\frac{n\pi}{L} x) \rangle}{\langle \cos(\frac{n\pi}{L} x)|\cos(\frac{n\pi}{L} x) \rangle}\\
                &=\frac{2}{L} \int_0^L x \cos(\frac{n\pi}{L} x) dx\\
                &=\frac{2}{L}\frac{L^2(\pi n\sin(\pi n)+\cos(\pi n)-1)}{\pi^2 n^2}\\
                &=\frac{2L(\pi n\sin(\pi n)+\cos(\pi n)-1)}{\pi^2 n^2}
            \end{align*}
            Which leaves us with a final family of eigenfunctions:
            \begin{equation*}
                \sum_{n=1}^\infty \frac{2L(\pi n\sin(\pi n)+\cos(\pi n)-1)}{\pi^2 n^2}e^{-k\frac{n^2\pi^2}{L^2} t}\cos(\frac{n\pi}{L} x)
            \end{equation*}

            \item To find the equilibrium solution we consider the case where the function $u$ has no dependence 
            on time, that is $\dot{u}(x,t)=0$. We can find this by solving
            \begin{equation*}
                \frac{d^2u_E}{dx^2}=0
            \end{equation*}
            and this is a full derivative as $u_E$ is a function of $x$ only. We can integrate this twice to find 
            \begin{align*}
                u_E(x)&=C_1x+C_2\\
                u_E'(x)&=C_1
            \end{align*}
            Looking at our BCs for this problem, we get
            \begin{align*}
                u_E(L)=0&=C_1L+C_2\\
                \implies C_1L&=-C_2\\
                u_E'(0)=0&=C_1\\
                \implies C_2&=0
            \end{align*}
            And I can't work out how to make this work.

        \end{enumerate}
    \end{enumerate}    

\end{document}
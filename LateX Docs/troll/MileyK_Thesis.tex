\documentclass[11pt]{article}
\usepackage[margin=1in, top=1in]{geometry}
\usepackage[all]{nowidow}
\usepackage[hyperfigures=true, hidelinks, pdfhighlight=/N]{hyperref}
\usepackage[separate-uncertainty=true, group-digits=false]{siunitx}
\usepackage{graphicx,amsmath,physics,tabto,float,amssymb,pgfplots,verbatim,tcolorbox}
\usepackage{listings,xcolor,subcaption,caption,import,wrapfig,minted}
\usepackage[sorting=none]{biblatex}
\usepackage[version=4]{mhchem}
\usepackage[noabbrev]{cleveref}
\usepackage[british]{babel}
\newcommand{\creflastconjunction}{, and\nobreakspace}
\newcommand{\mb}[1]{\mathbf{#1}}
\numberwithin{equation}{section}
\numberwithin{figure}{section}
\numberwithin{table}{section}
\captionsetup{font=small, belowskip=0pt}
\pgfplotsset{compat=1.17}
\addbibresource{bibliography.bib}
\usemintedstyle{vs}
\definecolor{LightGray}{HTML}{eaeaea}
\setminted{framesep=2mm,bgcolor=LightGray,fontsize=\footnotesize,linenos,breaklines}
\let\OldTexttt\texttt

\title{{\Huge A preliminary analysis of data from ALICE's new ITS and MFT detectors}}
\author{{\Large Miles Kidson}\\ \\
Supervisors: Prof. Zinhle Buthelezi, Dr. SV Fortsch, \& Prof. Tom Dietel\\
Assisted By: Dr. B Naik (Postdoctoral fellow)}
\date{\textbf{UCT Honours 2022}}

\begin{document}

\maketitle

\begin{figure}[h]
    \begin{center}
        \includegraphics{UCT.jpg}
    \end{center}
\end{figure}

\begin{abstract}
    \centering
    We investigate pilot-beam data from proton-proton collisions in Run 3 at ALICE, tracing its journey through two of the new detectors (the Inner Tracking System and Muon Forward Tracker) and the new Online-Offline analysis framework. The data is analysed to assess the performance of the detectors with the expectation of isotropic emission of particles, taking the detector geometry into account. Recommendations for improvements to the analysis framework and documentation are given.
\end{abstract}

\newpage

lol get jebaited loser

\end{document}
\documentclass[11pt]{article}
\usepackage[margin=.8in, top=0.3in]{geometry}
\usepackage[all]{nowidow}
\usepackage[hyperfigures=true, hidelinks, pdfhighlight=/N]{hyperref}
\usepackage[separate-uncertainty=true, group-digits=false]{siunitx}
\usepackage{graphicx,amsmath,physics,tabto,float,amssymb,pgfplots,verbatim,tcolorbox}
\usepackage{listings,xcolor,subfig,caption,import,wrapfig,lipsum}
\usepackage[version=4]{mhchem}
\numberwithin{equation}{section}
\numberwithin{figure}{section}
\numberwithin{table}{section}
\definecolor{stringcolor}{HTML}{C792EA}
\definecolor{codeblue}{HTML}{2162DB}
\definecolor{commentcolor}{HTML}{4A6E46}
\captionsetup{font=small, belowskip=0pt}
\lstdefinestyle{appendix}{
    basicstyle=\ttfamily\footnotesize,commentstyle=\color{commentcolor},keywordstyle=\color{codeblue},
    stringstyle=\color{stringcolor},showstringspaces=false,numbers=left,upquote=true,captionpos=t,
    abovecaptionskip=12pt,belowcaptionskip=12pt,language=Python,breaklines=true,frame=single}
\lstdefinestyle{inline}{
    basicstyle=\ttfamily\footnotesize,commentstyle=\color{commentcolor},keywordstyle=\color{codeblue},
    stringstyle=\color{stringcolor},showstringspaces=false,numbers=left,upquote=true,frame=tb,
    captionpos=b,language=Python}
\renewcommand{\lstlistingname}{Appendix}
\pgfplotsset{compat=1.17}

\begin{document}

\begin{center}
    {\huge Measuring the half-life of \ce{^{28}Al}}\\
    \vspace{0.2in}
    \textbf{KDSMIL001 | June 2021}
\end{center}

\section{Introduction}\label{sec:Introduction}
\par In this report we aim to determine the half-life of \ce{^{28}Al}. This isotope is not naturally occurring so in order to produce it we bombarded a cylinder of \ce{^{27}Al} with neutrons from a 2.2GBq americium-beryllium (AmBe) neutron source. These neutrons were moderated by water, hopefully slowing them enough to be thermal neutrons ($\lesssim 1$ MeV) at which point they have a larger interaction cross section with the \ce{^{27}Al} nucleus. Once the \ce{^{27}Al} has been `activated', i.e. a large proportion of the nuclei have absorbed a nucleus, (about 30 minutes in our case) we then placed a NaI(Tl) scintillator detector, connected to a high voltage power supply, pre-amplifier, amplifier, and multi-channel analyser with data acquisition software USX, inside the cylinder and recorded 180 10 second runs using the Multiple Runs feature. 
\par Taking 180 10 second runs as opposed to 1 30 minute run allows us to observe the change in total counts for a specific portion of the gamma ray spectrum for each time step and this should, provided we choose our portion properly, tell us about the half-life of \ce{^{28}Al}. This assumes that the number of \ce{^{28}Al} atoms in the cylinder is proportional to the number of gamma rays detected.

\section{Results and Analysis}\label{sec:Results}
\par Our first step in determining the half-life of \ce{^{28}Al} was determining which region to consider when trying to observe the decay of \ce{^{28}Al} in the source. \autoref{fig:Spectrum} shows the gamma ray spectrum for our activated \ce{^{27}Al} cylinder.

\begin{figure}[H]
    \begin{center}
       \subimport{Plots}{28AlSpectrum.pgf}
       \caption{Gamma ray spectrum for activated \ce{^{27}Al} made using a NaI scintillator detector at experimental station 1 in PHYLAB3. Energy calibration done using the same set-up by examining the primary photopeaks of \ce{^{137}Cs} \cite{137Cs}, \ce{^{22}Na} \cite{22Na}, and \ce{^{60}Co} \cite{60Co}. The 1778.987 keV photopeak is from the $\beta^-$ decay of \ce{^{28}Al} \cite{28Al}. The 843.76 and 1014.52 keV photopeaks both most likely come from a \ce{^{27}Al(n,p)^{27}Mg} interaction \cite{geneseo}, where the neutrons are too high energy to simply be absorbed and rather knock a proton out the nucleus and replace it. This \ce{^{27}Mg} then emits these two principle gamma rays when it decays through $\beta^-$ decay \cite{27Mg}. More details on Bremsstrahlung in Gilmore (2008).}
       \label{fig:Spectrum}
    \end{center}
\end{figure}
\par There are many features in the spectrum that aren't directly related to the decay of \ce{^{28}Al}, so they are not useful for measuring the half-life. What we did was fit a Gaussian function to the photopeak at 1778.987 keV, as we expect it to have a Gaussian form, and thus found $\sigma$ for the peak. We then chose the data within $\mu\pm3\sigma$ for the peak as that is very unlikely to include contributions from anything other than \ce{^{28}Al} and background \cite{geneseo}. We considered including contributions from the Compton continuum but in this case it isn't very well defined so we chose to leave it out.
\par We summed up the counts from all the channels within our range of interest for each 10 second run to find a decay curve. Now we expect that this decay curve will have the form of exponential decay, described by
\begin{equation}
    R(t)=R(0)e^{-\lambda t}+B\;\cite{manual}
    \label{eqn:exp decay}
\end{equation}
where $B$ is the level of background radiation. An initial approximation of the half-life by inspection found that it was around 2 minutes, which means that after 20 minutes, or 2/3 of the total experiment time, the \ce{^{28}Al} would be down to about 0.1\% of its original concentration, so by averaging the last 1/3 of the decay curve we can estimate the level of background radiation and subtract it from our data. 
\begin{wrapfigure}[18]{r}{0.5\textwidth}
    \begin{center}
        \scalebox{0.5}{\subimport{Plots}{28AlDecay.pgf}}    
        \caption{The linearised decay curve for \ce{^{28}Al} for 600s, data from between $\mu\pm3\sigma$ for the 1778.987 keV photopeak in \autoref{fig:Spectrum}. Best Fit Line calculated using a weighted linear least squares fit from Kirkup (2019).}
        \label{fig:decay}
    \end{center}
\end{wrapfigure}
\par Linearising the data and performing a weighted linear least squares fit \cite{kirkup} on the first 1/3 of the data to avoid dealing with negative or NaN values from when we subtracted background, we found $\lambda=\num{526.7(8.9)e-5}$. 
\par Using $t_{1/2}=\ln2/\lambda$ \cite{manual} we find a half life of $t_{1/2}=\SI{131.6(3.2)}{\second}=\SI{2.193(0.053)}{\minute}$. This does in fact agree with the value of \SI{2.245(0.002)}{\minute} stated in \cite{28Al} but in the interest of repeating measurements we used data taken by BNTVIC003 and ALMMOH008 and performed precisely the same analysis on it. Combining the results we get a half-life of $t_{1/2}=\SI{2.202(0.046)}{\minute}$.

\section{Conclusion}\label{sec:Conclusion}
The $\chi^2/$dof in \autoref{fig:decay}, as well as similar values for the other two datasets we analysed, give us confidence that the result $t_{1/2}=\SI{2.202(0.046)}{\minute}$ is accurate. This is again confirmed by its agreement within experimental uncertainty with the value in \cite{28Al}. The most notable improvement that could be made to the precision of our final result would be making sure the source contains as much \ce{^{28}Al} as possible through a longer time left exposed to the AmBe source. This, combined with better shielding against background radiation, is recommended.


\begin{thebibliography}{9}
    \bibitem{137Cs}
    E. Browne, J. K. Tuli, Nuclear Data Sheets 108, p2173 (2007)
    \bibitem{22Na}
    M. Shamsuzzoha Basunia, Nuclear Data Sheets 127, p69 (2015)
    \bibitem{60Co}
    E. Browne, J. K. Tuli, Nuclear Data Sheets 114, p1849 (2013)
    \bibitem{27Mg}
    M. Shamsuzzoha Basunia, Nuclear Data Sheets 112, p1875 (2011)
    \bibitem{28Al}
    M. Shamsuzzoha Basunia, Nuclear Data Sheets 114, p1189 (2013)
    \bibitem{geneseo}
    Nuclear Physics, Gamma Ray Spectra, 
    \texttt{https://www.geneseo.edu/nuclear/gamma-ray-spectra}
    \bibitem{manual}
    T. Hutton, \textit{PHY3004W Laboratory: Half-life experiment}, UCT Physics, 2021.
    \bibitem{kirkup}
    L. Kirkup, \textit{Experimental methods for science and engineering students}, (2019)
    
\end{thebibliography}

\end{document}
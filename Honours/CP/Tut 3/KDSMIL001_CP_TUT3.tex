\documentclass[11pt]{article}
\usepackage[margin=1in, top=1in]{geometry}
\usepackage[all]{nowidow}
\usepackage[hyperfigures=true, hidelinks, pdfhighlight=/N]{hyperref}
\usepackage[separate-uncertainty=true, group-digits=false]{siunitx}
\usepackage{graphicx,amsmath,physics,tabto,float,amssymb,pgfplots,verbatim,tcolorbox}
\usepackage{listings,xcolor,subfig,caption,import,wrapfig,enumitem}
\usepackage[version=4]{mhchem}
\usepackage[noabbrev]{cleveref}
\newcommand{\creflastconjunction}{, and\nobreakspace}
\definecolor{stringcolor}{HTML}{C792EA}
\definecolor{codeblue}{HTML}{2162DB}
\definecolor{commentcolor}{HTML}{4A6E46}
\captionsetup{font=small, belowskip=0pt}
\lstdefinestyle{appendix}{
    basicstyle=\ttfamily\footnotesize,commentstyle=\color{commentcolor},keywordstyle=\color{codeblue},
    stringstyle=\color{stringcolor},showstringspaces=false,numbers=left,upquote=true,captionpos=t,
    abovecaptionskip=12pt,belowcaptionskip=12pt,language=Python,breaklines=true,frame=single}
\lstdefinestyle{inline}{
    basicstyle=\ttfamily\footnotesize,commentstyle=\color{commentcolor},keywordstyle=\color{codeblue},
    stringstyle=\color{stringcolor},showstringspaces=false,numbers=left,upquote=true,frame=tb,
    captionpos=b,language=Python}
\renewcommand{\lstlistingname}{Appendix}
\pgfplotsset{compat=1.17}

\begin{document}

\begin{center}
    \textbf{CP Tut 3}\hspace{2in}\textbf{KDSMIL001}\hspace{2in}\textbf{07-05-2022}
\end{center}

\begin{enumerate}
    \item To use importance sampling with Monte Carlo integration, we identified that the integrand of 
    \begin{equation}
        I=\int_{-1}^6 dx \int_0^5 dy \int_0^5 dz \; e^{-x/3} \left(1+0.1\ln(\sqrt{x^2+y^2+z^2+1})\right)
    \end{equation}
    is mostly flat over the interval, in terms of $y$ and $z$ at least, so we can sample those values just by a uniform distribution over the interval. The $x$ values need to be sampled according to $e^{-x/3}$, using the inverse transform method. To do this we consider $w_i$'s distributed uniformly on $[0,1]$. Then
    \begin{align*}
        w_i&=\int_{-\infty}^{x_i}e^{-x'/3}dx'\\
        &=\int_{-1}^{x_i}e^{-x'/3}dx'\\
        &=-3(e^{-x_i/3}-e^{-1/3})\\
        \implies x_i&=-3\ln\left(-\frac{w_i}{3}+e^{1/3}\right)
    \end{align*}
    gives us the distribution of $x$ values.\\
    Now we want to calculate $I$ using 
    \begin{equation}
        I=\frac{1}{N}\sum_{i=1}^N \frac{f(x_i, y_i, z_i)}{w(x_i, y_i, z_i)}.
        \label{eqn:q1Int}
    \end{equation}
    In this case $f(x,y,z)$ is the original integrand. $w(x,y,z)$ needs to be normalised, so can't just be $e^{-x/3}$, it needs to be divided by the product of the integral of it over the integration interval, including the $y$ and $z$ intervals, which turns out to be $-3(e^{-2}-e^{1/3})\times 25$.\\
    Generating $x_i$ values using the inverse transform method, and $y_i$ and $z_i$ uniformly, we used \cref{eqn:q1Int} and generated \num[]{1000000} points per coordinate, finding $I=107.11640605103929$.
\end{enumerate}

\end{document}
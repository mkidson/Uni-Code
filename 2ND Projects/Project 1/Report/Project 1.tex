\documentclass[12pt]{article}
\usepackage[margin=1.2in]{geometry}
\usepackage[all]{nowidow}
\usepackage[hyperfigures=true, hidelinks, pdfhighlight=/N]{hyperref}
\usepackage[separate-uncertainty=true,group-digits=false]{siunitx}
\usepackage{graphicx,amsmath,physics,tabto,float,amssymb,pgfplots,verbatim,tcolorbox}
\usepackage{listings,xcolor,subfig,keyval2e,caption,import}
\numberwithin{equation}{section}
\numberwithin{figure}{section}
\numberwithin{table}{section}
\definecolor{stringcolor}{HTML}{C792EA}
\definecolor{codeblue}{HTML}{2162DB}
\definecolor{commentcolor}{HTML}{4A6E46}
\lstdefinestyle{appendix}{
    basicstyle=\ttfamily\footnotesize,commentstyle=\color{commentcolor},keywordstyle=\color{codeblue},
    stringstyle=\color{stringcolor},showstringspaces=false,numbers=left,upquote=true,captionpos=t,
    abovecaptionskip=12pt,belowcaptionskip=12pt,language=Python,breaklines=true,frame=single}
\lstdefinestyle{inline}{
    basicstyle=\ttfamily\footnotesize,commentstyle=\color{commentcolor},keywordstyle=\color{codeblue},
    stringstyle=\color{stringcolor},showstringspaces=false,numbers=left,upquote=true,frame=tb,
    captionpos=b,language=Python}
\renewcommand{\lstlistingname}{Appendix}
\pgfplotsset{compat=1.17}

\title{Chaotic Ingredients}
\author{KDSMIL001 \; MAM2046W 2ND}
\date{\textbf{10 November 2020}}

\begin{document}
    \maketitle
    
    \section{Introduction}\label{sec:Introduction}
    The purpose of this project is to investigate one of the simplest system in which chaotic 
    behaviour can be observed. We know from the Poincar\'{e}-Bendixon theorem that we must 
    be looking at a 3-dimensional system in order to have chaos, however to illustrate our 
    point we begin in 2-D.

    \section{A 2-dimensional Prologue}\label{sec:Prologue}
    We consider the following system:
    \begin{align*}
        \dot x&=-y\\
        \dot y&-x+ay
    \end{align*}
    where we have $a=0.1$. This system is clearly represented by 
    \begin{equation}
        \begin{pmatrix}
            \dot x \\
            \dot y
        \end{pmatrix}=
        \underbrace{
        \begin{pmatrix}
            0 & -1\\
            1 & a
        \end{pmatrix}}_{Jacobian}
        \begin{pmatrix}
            x \\
            y
        \end{pmatrix}
        \label{eqn:System 1}
    \end{equation}
    The system clearly has a fixed point at the origin and we can analyse the stability of that 
    fixed point by looking at the Jacobian. The trace is $\tau=a>0$ and the determinant is 
    $\Delta=1>0$, meaning our fixed point is an unstable spiral. \autoref{fig:Q1} is a plot of 
    a trajectory starting at $(0.3,0.3)$.
    \begin{figure}[H]
        \begin{center}
           \scalebox{.7}{\subimport{Plots}{Q1.pgf}}
           \caption{A trajectory of the system in \autoref{eqn:System 1} starting from $(0.3,0.3)$}
           \label{fig:Q1}
        \end{center}
    \end{figure}
    
    \section{A 3-dimensional Linear System}\label{sec:3D Linear}
    As stated before, we require 3 dimensions to even be able to observe chaotic behaviour, so 
    let us add on to our original system the following:
    \begin{equation*}
        \dot z=a-bz
    \end{equation*}
    where $b$ is some positive parameter that we can vary. We can notice that $z$ is decoupled 
    from the other two coordinates, and that it's stable, so it will return to its fixed point 
    as time goes on, that fixed point being easily found to be $(0,0,\frac{a}{b})$. To confirm 
    this stability in $z$, we look at the Jacobian:
    \begin{equation}
        J=
        \begin{pmatrix}
            0 & -1 & 0\\
            1 & a & 0\\
            0 & 0 & -b
        \end{pmatrix}
        \label{eqn:3D Linear Jacobian}
    \end{equation}
    The eigenvalues of this Jacobian are 
    \begin{equation*}
        \lambda_{1,2}=\frac{1}{2}(0.1\pm\sqrt{0.1^2-4}),\;\; \lambda_3=-b
    \end{equation*}
    Since $b$ is a positive parameter, we can see that $\lambda_3<0$, so the eigenvector 
    associated with it is stable, while the other two eigenvectors are unstable, as their 
    real part is $>0$. The eigenvector associated with $\lambda_3$ is 
    \begin{equation*}
        \begin{pmatrix}
            0\\0\\1
        \end{pmatrix}
    \end{equation*}
    so $z$ will always return to its fixed point, regardless of the behaviour of $x$ or $y$. 
    To visualise this, we have plotted the trajectories from random initial conditions near to 
    the fixed point for various values of $b$, and it's clear to see that $z$ always returns 
    to its fixed point $\frac{a}{b}$.
    \begin{figure}[H]%
        \centering
        \subfloat[$b=0.01$]{\scalebox{0.4}{\subimport{Plots}{Q2_0.01.pgf}}}
        \,
        \subfloat[$b=0.1$]{\scalebox{0.4}{\subimport{Plots}{Q2_0.1.pgf}}}
        \,
        \subfloat[$b=1$]{\scalebox{0.4}{\subimport{Plots}{Q2_1.pgf}}}
        \,
        \subfloat[$b=10$]{\scalebox{0.4}{\subimport{Plots}{Q2_10.pgf}}}
        \caption{Trajectories starting from the same random initial conditions near to the 
        origin, for the system in \autoref{eqn:3D Linear Jacobian}, for various values of $b$.}
        \label{fig:Q2}
    \end{figure}
    As can be see from the plots in \autoref{fig:Q2}, any initial condition near to the origin 
    will lead to an unstable spiral around the origin of the $xy$-plane, with the $z$ coordinate 
    returning to $\frac{a}{b}$. For smaller $b$, the fixed point of $z$ is further from the origin, 
    so it takes longer to get there. 

    \section{A 3-dimensional Nonlinear System}\label{sec:3D Nonlinear}
    Clearly just adding a third dimension doesn't automatically give us chaotic behaviour, it's 
    a little bit more complicated than that. We actually need to have all three equations coupled, 
    and feeding back into each other, so we change the system to
    \begin{align*}
        \dot x&=-y-z\\
        \dot y&=x+ay\\
        \dot z&=a+z(x-b)
    \end{align*}
    This means that as $x$ grows, so will z, leading to some lovely feedback. As always we want 
    to examine this system to understand its fixed points and their stability. We can find the 
    fixed points to be 
    \begin{align*}
        x&=\frac{b\pm\sqrt{b^2-0.04}}{2}\\
        y&=-\frac{b\pm\sqrt{b^2-0.04}}{0.2}\\
        z&=\frac{b\pm\sqrt{b^2-0.04}}{0.2}
    \end{align*}
    which we will call $\vec x_+$ and $\vec x_-$. The Jacobian of the system is
    \begin{equation}
        J=
        \begin{pmatrix}
            0 & -1 & -1\\
            1 & a & 0 \\
            z & 0 & x-b
        \end{pmatrix}
        \label{eqn:3D Nonlinear Jacobian}
    \end{equation}
    Expressing the eigenvalues of this matrix with respect to $b$ is extremely messy, so we have 
    done you the service of calculating them for a wide range of values of $b$ in \autoref{tbl:Eigenvalues}.
    \begin{table}[H]
        \centering
        \begin{tabular}{c|c c}
            $b$ & $\vec x_+$ & $\vec x_-$\\
            \hline
            $0.01$ & $0.451-1.12j$ & $0.451+1.12j$\\
             & $-0.451+1.12j$ & $-0.451-1.12j$\\
             & $0.0951+0.099j$ & $0.0951-0.099j$\\
            $0.02$ & $0.442-1.14j$ & $0.442+1.14j$\\
             & $-0.442+1.14j$ & $-0.442-1.14j$\\
             & $0.0902+0.0986j$ & $0.0902-0.0986j$\\
            $0.1$ & $0.344-1.27j$ & $0.344+1.27j$\\
             & $-0.345+1.27j$ & $-0.345-1.27j$\\
             & $0.0505+0.0863j$ & $0.0505-0.0863j$\\
            $0.2$ & $1.41j$ & $1.41j$\\
             & $-1.41j$ & $-1.41j$\\
             & $\num{-4.79e-19}$ & $\num{-4.79e-19}$\\
            $1$ & $\num{-4.16e-06}+3.30j$ & $0.0234+1.02j$\\
             & $\num{-4.16e-06}-3.30j$ & $0.0234-1.02j$\\
             & $\num{8.99e-02}$ & $-0.937$\\
            $2$ & $\num{-1.14e-06}+4.56j$ & $0.0398+1.003j$\\
             & $\num{-1.14e-06}-4.56j$ & $0.0398-1.003j$\\
             & $\num{9.5e-02}$ & $-1.97$\\
            $10$ & $\num{-4.9e-08}+10.05j$ & $0.0495+0.999j$\\
             & $\num{-4.9e-08}-10.05j$ & $0.0495-0.999j$\\
             & $\num{9.9e-02}$ & $-9.9$\\
            $20$ & $\num{-1.24e-08}+14.2j$ & $0.0499+0.999j$\\
             & $\num{-1.24e-08}-14.2j$ & $0.0499-0.999j$\\
             & $\num{9.95e-02}$ & $-19.9$\\
        \end{tabular}
        \caption{Eigenvalues of the system in \autoref{eqn:3D Nonlinear Jacobian} for various 
        values of $b$.}
        \label{tbl:Eigenvalues}
    \end{table}
    Note that these values have been rounded in order to be displayed easily but this doesn't 
    affect our analysis, since all we are interested in is the sign of the real part of the 
    eigenvalues as that tells us the stability of the respective eigenvector. What we can see 
    from the eigenvalues is that for $b<0.2$ the real part of the first two eigenvalues for a 
    given fixed point have opposite sign but equal magnitude, which is some sort of unstable 
    spiral. For $b>0.2$ the two eigenvalues have the same real part for a given fixed point, 
    indicating a stable spiral. For $b=0.2$ we expect stable orbits, at least in some sense. 
    Any instability will come from the third eigenvalue. Speaking of, the third eigenvalue is 
    always different in magnitude from the first two, but we notice that for $b<0.2$ it always 
    has positive real part, indicating instability for both fixed points. For $b>0.2$ we see 
    that it is stable for one fixed point and unstable for the other. 
    \newline
    \newline
    Now we get to the exciting part, visualising some lovely trajectories. 
    \begin{figure}[H]%
        \centering
        \subfloat[$b=0.01$]{\scalebox{0.4}{\subimport{Plots}{Q3_0.01.pgf}}}
        \,
        \subfloat[$b=0.02$]{\scalebox{0.4}{\subimport{Plots}{Q3_0.02.pgf}}}
        \,
        \subfloat[$b=0.1$]{\scalebox{0.4}{\subimport{Plots}{Q3_0.1.pgf}}}
        \,
        \subfloat[$b=0.2$]{\scalebox{0.4}{\subimport{Plots}{Q3_0.2.pgf}}}
        \,
        \subfloat[$b=1$]{\scalebox{0.4}{\subimport{Plots}{Q3_1.pgf}}}
        \,
        \subfloat[$b=2$]{\scalebox{0.4}{\subimport{Plots}{Q3_2.pgf}}}
        \,
        \subfloat[$b=10$]{\scalebox{0.4}{\subimport{Plots}{Q3_10.pgf}}}
        \,
        \subfloat[$b=20$]{\scalebox{0.4}{\subimport{Plots}{Q3_20.pgf}}}
        \caption{Trajectories starting from the same random initial conditions near to the 
        origin, for the system in \autoref{eqn:3D Nonlinear Jacobian}, for various values of $b$.}
        \label{fig:Q3}
    \end{figure}
    \autoref{fig:Q3} shows the trajectories of the systems we found eigenvalues for earlier. We 
    can clearly see that for $b>0.2$, everything looks nice and regular with $z$ collapsing to 
    its fixed point pretty quickly and then leading to unstable spirals around the origin of 
    the $xy$-plane. For $b<0.2$ however, things get a little crazy, with chaos-like behaviour 
    being seen. The reason that we attribute to this behaviour is the fact that, for $b<0.2$, 
    the fixed points of the system become complex, which leads to all sorts of mishaps and mistakes. 

    \section{Conclusion}\label{sec:Conclusion}
    We found that simply introducing a third dimension doesn't automatically create a chaotic 
    system. This makes sense as pretty much all systems in the real world are three dimensional 
    and don't devolve into chaos immediately. In fact they take millions of years to do it 
    properly. Either way, it takes some careful consideration of the constructions of the system 
    to allow for chaos, and even then it only seems to arise for, in our case, some values of $b$. 


\end{document}
\documentclass[12pt]{article}
\usepackage[margin=1.2in]{geometry}
\usepackage{graphicx}
\usepackage{amsmath}
\usepackage{physics}
\usepackage{tabto}
\usepackage{float}
\usepackage{amssymb}
\usepackage{pgfplots}
\usepackage{verbatim}
\usepackage{tcolorbox}
\usepackage{listings}
\usepackage{xcolor}
\usepackage{siunitx}
\definecolor{stringcolor}{HTML}{C792EA}
\definecolor{codeblue}{HTML}{2162DB}
\definecolor{commentcolor}{HTML}{4A6E46}
\lstdefinestyle{appendix}{
    basicstyle=\ttfamily\footnotesize,
    commentstyle=\color{commentcolor},
    keywordstyle=\color{codeblue},
    stringstyle=\color{stringcolor},
    showstringspaces=false,
    numbers=left,
    upquote=true,
    captionpos=t,
    abovecaptionskip=12pt,
    belowcaptionskip=12pt,
    language=Python,
    breaklines=true,
    frame=single
}
\lstdefinestyle{inline}{
    basicstyle=\ttfamily\footnotesize,
    commentstyle=\color{commentcolor},
    keywordstyle=\color{codeblue},
    stringstyle=\color{stringcolor},
    showstringspaces=false,
    numbers=left,
    upquote=true,
    frame=tb,
    captionpos=b,
    language=Python
}
\renewcommand{\lstlistingname}{Appendix}
\pgfplotsset{compat=1.16}

\title{Assignment 1}
\date{4 March 2020}
\author{}

\begin{document}

    \begin{titlepage}
        \maketitle
        \center
        \textbf{\large{MAM2046W 2NA}}\\[12pt]
        \textbf{\large{KDSMIL001}}\\
    \end{titlepage}

    \begin{enumerate}
        \item Newton-Raphson root-finding
        \begin{enumerate}
            \item We are to find $f(x)$. In the Newton-Raphson Method, the formula is 
            \begin{equation*}
                x_{n+1} = x_n - \frac{f(x)}{f'(x)}
            \end{equation*}
            So, we can deduce from the equation given to us
            \begin{equation*}
                x_{n+1} = x_n - (\cos x_n)(\sin x_n) + R\cos^2 x_n
            \end{equation*}
            that
            \begin{equation*}
                \begin{split}
                    &\frac{f(x)}{f'(x)} = \cos x_n \sin x_n + R\cos x_n \\
                    \implies &\frac{df}{f} = \frac{1}{\cos x_n \sin x_n + R\cos x_n}dx \\
                \end{split}
            \end{equation*}
            Now, we perform a substitution in order to simplify some things. Let $x_n = \arctan u$. Then
            \begin{equation*}
                \begin{split}
                    &dx = \frac{1}{1+u^2}du \\
                    \implies &\int\frac{df}{f} = \int\frac{1}{\cos x_n \sin x_n + R\cos x_n}dx \\
                    \implies &\ln|f| = \int\frac{1}{\frac{u}{\sqrt{1+u^2}}\frac{1}{\sqrt{1+u^2}} + \frac{R}{1+u^2}}\frac{1}{1+u^2}du \\
                    \implies &\ln|f| = \int\frac{1}{\frac{u}{1+u^2} + \frac{R}{1+u^2}}\frac{1}{1+u^2}du \\
                    \implies &\ln|f| = \int\frac{1}{u + R}du \\
                    \implies &\ln|f| = \ln(u + R) + C, \hspace{15pt} C \in \Re \\
                    \implies &f = (u + R)e^C
                \end{split}
            \end{equation*}
            Let $e^C = A$, then 
            \begin{equation*}
                f = A(\tan x_n + R)
            \end{equation*}
            \item This formula can be used to find the roots of some function $f$.
        \end{enumerate}
        \item Fixed Point Iteration
        \begin{enumerate}
            \item Given the equation of the form $f(x)=x^2-x-2=0$, it doesn't take much manipulation to 
            reduce it to the first two solutions:
            \begin{equation*}
                g_1(x) = x^2-2
            \end{equation*}
            \begin{equation*}
                g_2(x) = \sqrt{x+2}
            \end{equation*}
            For the next two solutions, a little bit more manipulation is required:
            \begin{equation*}
                \begin{split}
                    &x^2-x-2=0 \\
                    \implies &x^2-x=2 \\
                    \implies &x(x-1)=2 \\
                    \implies &x=\frac{2}{x-1} \\
                    \implies &g_3(x) = \frac{2}{x-1}
                \end{split}
            \end{equation*}
            and
            \begin{equation*}
                \begin{split}
                    &x^2-x-2=0 \\
                    \implies &x^2-x=2 \\
                    \implies &x(x-1)=2 \\
                    \implies &x-1=\frac{2}{x} \\
                    \implies &x=\frac{2}{x}+1 \\
                    \implies &g_4(x)=\frac{2}{x}+1
                \end{split}
            \end{equation*}

            \item To plot these functions on the same axes, I used Python with the \\ 
            \texttt{matplotlib} module, below is the result [Figure \ref{fig:Q2 Plot}].
            It's clear to see that all of the graphs intersect the $y=x$ line at two points with the 
            exception of $g_2(x)$, which has no values $<0$ as a function. The code for this can be 
            found in [Appendix 1]

            \begin{figure}[H]
                \begin{center}
                   \scalebox{.8}{\input{Q2 Plot.pgf}}
                   \caption{Plot of each $g_i(x)$}
                   \label{fig:Q2 Plot}
                \end{center}
            \end{figure}
            \item As we can see from the plot, both $g'_2(x)$ and $g'_4(x)$ are less than 1, which means 
            they will converge on the fixed point. The other two, however, will not converge on the 
            positive root. We can't say whether they will diverge or whether they will converge on the 
            negative root, but they will not converge on the positive root, which is what we are looking for. 
        \end{enumerate}
        \item Halley's Method
            \begin{enumerate}
                \item The program below, [Q3a.py], results in a value of $x_n = 2.154434690031884$ after 4 iterations, 
                excluding the initial guess. As you can see, the stop condition for this method was that the 
                difference between $x_n$ and $x_{n+1}$ has an absolute value less than $1\times10^{-12}$. This 
                method is not guaranteed to work for every function, but in this case it was sufficient. 
                \newline
                \newline
                \begin{minipage}{\linewidth}
                    \lstinputlisting[title=Q3a.py, style=inline]{Q3a.py}
                \end{minipage}
                
                \item Comparing the number of iterations for Halley's Method to the number of iterations 
                needed to reach a root when using Newton's Method, which in this case would be 5 iterations 
                excluding the initial guess, we can see that while Halley's Method is better, it's not by much. 
                The code performing this calculation is below [Q3b.py]. It also returns $x_n = 2.154434690032$.
                \newline
                \newline
                \begin{minipage}{\linewidth}
                    \lstinputlisting[title=Q3b.py, style=inline]{Q3b.py}
                \end{minipage}
                
                \item Firstly, to check that there is a root in the interval we can use the Intermediate 
                Value Theorem along with the fact that $f(2)=-2$ and $f(4)=6$. Next we can write a program, 
                [Q3c.py] below, which does the iterations for us.
                Comparing the result in Problem 3a to the result obtained when using the 
                Bisection method, calculated below in [Q3c.py], we can see that Bisection converges 
                in 40 iterations. Compared to the 3 iterations that Halley's Method took, this is terrible. 
                The upside to this method is that it is guaranteed to find the root, as long as the root exists 
                in the interval.
                \newline
                \newline
                \begin{minipage}{\linewidth}
                    \lstinputlisting[title=Q3c.py, style=inline]{Q3c.py}
                \end{minipage}
            \end{enumerate}
        \item Firstly, the functions we are looking at are
        \begin{equation*}
            \begin{split}
                &g_1(x)=x^2-2 \\
                &g_2(x)=\sqrt{x+2} \\
                &g_3(x)=\frac{2}{x-1} \\
                &g_4(x)=\frac{2}{x}+1 \\
            \end{split}
        \end{equation*}
        
        Using these functions, the code below uses Fixed Point Iteration to find one of the roots of the 
        original function $f(x) = x^2 - x - 2$. We hoped to find the positive root that was visible 
        in Figure \ref{fig:Q2 Plot}, but after running the program we found these results:

        \begin{table}[H]
            \centering
            \begin{tabular}{ccc}
                \hline
                Function & Fixed Point & \# of Iterations \\
                \hline
                $g_1(x)$ & Could not converge & 100 \\
                $g_2(x)$ & 2 & 7 \\
                $g_3(x)$ & -1 & 21 \\
                $g_4(x)$ & 2 & 16 \\
                \hline
            \end{tabular}
        \end{table}

        These values are all rounded as the program ran to an accuracy of \num{1e-12}, which returned values 
        such as 1.9999999999995248, which are unnecessarily verbose. 
        \newline
        On the topic of convergence to a root, all but one of the functions converged to a root, but 
        $g_3(x)$ seemed to converge on the negative root, namely -1. $g_1(x)$ did not converge, which was 
        expected as $|g'_1(x)| > 1$ at the root. The same was true for $g_3(x)$, but its derivative was 
        negative at the root, which led to it converging on the negative root. This is consistent with our 
        prediction in Question2a but it also shows us which function converges on the negative root and 
        which diverges. 
        \newline
        \lstinputlisting[title=Q4.py, style=inline]{Q4.py}
        
        \newpage
        \section*{Appendix}
        \lstinputlisting[caption=Q2.py, style=appendix]{Q2.py}
        
        
    \end{enumerate}

\end{document}
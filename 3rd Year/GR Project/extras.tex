% We note that for neighbouring points we have 
% \begin{align*}
%     dr&=\frac{\partial r}{\partial x}dx + \frac{\partial r}{\partial y}dy + \frac{\partial r}{\partial z}dz\\
%     d\theta&=\frac{\partial \theta}{\partial x}dx + \frac{\partial \theta}{\partial y}dy + \frac{\partial \theta}{\partial z}dz\\
%     d\phi&=\frac{\partial \phi}{\partial x}dx + \frac{\partial \phi}{\partial y}dy + \frac{\partial \phi}{\partial z}dz\\
% \end{align*}
% The calculation of the derivatives takes a bit of work, but it isn't too hard to see that it simplifies to 
% \begin{align*}
%     dr&=\sin\theta\cos\phi dx + \sin\theta\sin\phi dy + \cos\theta dz\\
%     d\theta&=\frac{\cos\theta\cos\varphi}{r}dx + \frac{\cos\theta\sin\varphi}{r} dy -\frac{\sin\theta}{r}dz\\
%     d\phi&=-\frac{\sin\varphi}{r\sin\theta}dx + \frac{\cos\varphi}{r\sin\theta}dy + 0dz
% \end{align*}
% This gives us our components of the $\Lambda_\beta^{\bar\alpha}$ matrix:


% where we have 
% \begin{align*}
%     \frac{\partial \varphi}{\partial r}&=\frac{\partial\varphi}{\partial x}\frac{\partial x}{\partial r}+\frac{\partial\varphi}{\partial y}\frac{\partial y}{\partial r}+\frac{\partial\varphi}{\partial z}\frac{\partial z}{\partial r}\\
%     &=\sin\theta\cos\phi\frac{\partial\varphi}{\partial x}+\sin\theta\sin\phi\frac{\partial\varphi}{\partial y}+\cos\theta\frac{\partial\varphi}{\partial z}\\
%     \frac{\partial \varphi}{\partial \theta}&=\frac{\partial\varphi}{\partial x}\frac{\partial x}{\partial \theta}+\frac{\partial\varphi}{\partial y}\frac{\partial y}{\partial \theta}+\frac{\partial\varphi}{\partial z}\frac{\partial z}{\partial \theta}\\
%     &=r\cos\theta\cos\phi\frac{\partial\varphi}{\partial x}+r\cos\theta\sin\phi\frac{\partial\varphi}{\partial y}-r\sin\theta\frac{\partial\varphi}{\partial z}\\
%     \frac{\partial \varphi}{\partial \phi}&=\frac{\partial\varphi}{\partial x}\frac{\partial x}{\partial \phi}+\frac{\partial\varphi}{\partial y}\frac{\partial y}{\partial \phi}+\frac{\partial\varphi}{\partial z}\frac{\partial z}{\partial \phi}\\
%     &=-r\sin\theta\sin\phi\frac{\partial\varphi}{\partial x}+r\sin\theta\cos\phi\frac{\partial\varphi}{\partial y}+0{\partial\varphi}{\partial z}
% \end{align*}
% The one-form transforms like 
% \begin{equation*}
%     \frac{\partial\varphi}{\partial x^{\bar\beta}}=\Lambda_{\bar\beta}^\alpha\frac{\partial\varphi}{\partial x^\alpha}
% \end{equation*}

% so we can construct $\Lambda_{\bar\beta}^\alpha$:
% \begin{equation}
%     \Lambda_{\bar\beta}^\alpha=
%     \begin{pmatrix}
%         \sin\theta\cos\varphi&r\cos\theta\cos\varphi&-r\sin\theta\sin\varphi\\
%         \sin\theta\sin\varphi&r\cos\theta\sin\varphi&r\sin\theta\cos\varphi\\
%         \cos\theta&-r\sin\theta&0
%     \end{pmatrix}
% \end{equation}

% \par Before we get into anything to do with wave equations, let us calculate some things that we will need. First off, the inverse metric is fairly simple to find. For the $rr$ and $\theta\theta$ components we can simply invert them as we did with the spherical polar coordinates metric. This comes about simply from linear algebra as those terms have no off-diagonal elements that interfere. The other components are a different story, but they are also fairly easy to deal with as we can represent them as a $2\times2$ matrix, for which finding the inverse is simply given by 
% \begin{equation*}
%     \begin{pmatrix}
%         a & b \\
%         c & d
%     \end{pmatrix}^{-1}
%     =
%     \frac{1}{ad-bc}
%     \begin{pmatrix}
%         d & -b \\
%         -c & a
%     \end{pmatrix}
% \end{equation*}
% where we are considering the matrix
% \begin{equation*}
%     \tilde{g}_{\alpha\beta}=
%     \begin{pmatrix}
%         (-1+\frac{2mr}{\Sigma}) & \frac{-2amr\sin^2\theta}{\Sigma} \\
%         \frac{-2amr\sin^2\theta}{\Sigma} & \left(r^2+a^2+\frac{2mra^2\sin^2\theta}{\Sigma}\right)\sin^2\theta
%     \end{pmatrix}
% \end{equation*}
% This has determinant $-\Delta\sin^2\theta$, so the inverse is given by
% \begin{align*}
%     \tilde{g}^{\alpha\beta}&=\frac{1}{-\Delta\sin^2\theta}
%     \begin{pmatrix}
%         \left(r^2+a^2+\frac{2mra^2\sin^2\theta}{\Sigma}\right)\sin^2\theta & \frac{2amr\sin^2\theta}{\Sigma} \\
%         \frac{2amr\sin^2\theta}{\Sigma} & (-1+\frac{2mr}{\Sigma})
%     \end{pmatrix}\\
%     &=
%     \begin{pmatrix}
%         -\frac{1}{\Delta}\left(r^2+a^2+\frac{2mra^2\sin^2\theta}{\Sigma}\right) & \frac{-2mra}{\Sigma\Delta} \\
%         \frac{-2mra}{\Sigma\Delta} & \frac{\Delta-a^2\sin^2\theta}{\Sigma\Delta\sin^2\theta}
%     \end{pmatrix}
% \end{align*}
% We can now construct the inverse of the Kerr-Newman metric by returning the $rr$ and $\theta\theta$ components to the matrix above after inverting them:

% \begin{align*}
    %     g=det(g_{\alpha\beta})=&(-1+\frac{2mr}{\Sigma})\left(\Sigma\left(\frac{\Sigma}{\Delta}\sin^2\theta\left(r^2+a^2+\frac{2mra^2\sin^2\theta}{\Sigma}\right)-0\right)\right)\\&-0+0-\left(\frac{-2amr\sin^2\theta}{\Sigma}\left(0-\frac{\Sigma}{\Delta}\left(0-\Sigma\frac{-2amr\sin^2\theta}{\Sigma}\right)\right)\right)\\
    %     =&\frac{-\Sigma^2\sin^2\theta}{\Delta}\left[\left(a^2+r^2+\frac{2amr\sin^2\theta}{\Sigma}\right)+\left(\frac{2mra^2}{\Sigma}+\frac{2mr^3}{\Sigma}\right)\right]\\
    %     =&\frac{-\Sigma^2\sin^2\theta}{\Delta}\left[a^2+r^2-2mr\frac{r^2+a^2\cos^2\theta}{\Sigma}\right]\\
    %     =&-\Sigma^2\sin^2\theta
% \end{align*}
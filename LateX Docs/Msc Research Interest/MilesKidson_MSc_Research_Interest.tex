\documentclass[11pt]{article}
\usepackage[margin=1in, top=1in]{geometry}
\usepackage[all]{nowidow}
\usepackage[hyperfigures=true, hidelinks, pdfhighlight=/N]{hyperref}
\usepackage[separate-uncertainty=true, group-digits=false]{siunitx}
\usepackage{graphicx,amsmath,physics,tabto,float,amssymb,pgfplots,verbatim,tcolorbox}
\usepackage{listings,xcolor,subfig,caption,import,wrapfig,minted,biblatex}
\usepackage[version=4]{mhchem}
\usepackage[noabbrev]{cleveref}
\usepackage[british]{babel}
\newcommand{\creflastconjunction}{, and\nobreakspace}
\newcommand{\mb}[1]{\mathbf{#1}}
\numberwithin{equation}{section}
\numberwithin{figure}{section}
\numberwithin{table}{section}
\captionsetup{font=small, belowskip=0pt}
\pgfplotsset{compat=1.17}
\addbibresource{bibliography.bib}
\usemintedstyle{vs}
\definecolor{LightGray}{HTML}{eaeaea}
\setminted{framesep=2mm,bgcolor=LightGray,fontsize=\footnotesize,linenos,breaklines}
\let\OldTexttt\texttt
\sethlcolor{LightGray}
\renewcommand{\texttt}[1]{\OldTexttt{\hl{#1}}}

\begin{document}

\begin{center}
    {\Large Research Interest for MSc for Miles Kidson}

\end{center}

Over the year of 2022 I have been working with iThemba LABS in collaboration with ALICE at CERN, looking at some of the detectors that were upgraded for the latest run of data taking at the LHC. In the process I have learned a lot about the practical side of experimental particle physics, such as how huge amounts of data are handled, processed, and analysed, how large scale detector systems work together, and more particularly how the new analysis system works at ALICE.

I am interested in continuing working in the ALICE group, either extending my current work on the Muon Forward Tracker and Inner Tracking System, or working in Assoc. Prof. Tom Dietel's team at UCT, working on the Transition Radiation Detector system at ALICE. With the increased energy and luminosity of Run 3, there is a lot of work to be done to fully take advantage of the detectors available to us.  

\end{document}
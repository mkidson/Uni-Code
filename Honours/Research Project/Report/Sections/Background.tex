\subsection{The ALICE Detector}
\begin{itemize}
    \item What is the LHC?
    \item What is ALICE?
    \item What does ALICE look for?
    \item What is Run 3?
\end{itemize}
The ALICE detector (A Large Ion Collider Experiment) is a detector experiment at the Large Hadron Collider (LHC) at CERN. Its primary goal is the investigation of ``strongly interacting matter at extreme energy densities, where a formation of a new phase of matter, the quark-gluon plasma, is expected'' (\cite{ALICE_LOI}). It achieves this goal by studying the products of head-on collisions of heavy ions such as lead. 

% ALICE is situated at 

% The LHC at CERN in Geneva is built to accelerate particles up to very high energies (\SI{13.6}{\tera\electronvolt})

The coordinate system used at ALICE needs to be discussed first in order to fully explain the scope of this report. A modified cylindrical coordinate system is used as most detectors in the experiment are cylindrically symmetric about the beamline of the LHC. We place the $z$-axis along the beamline and call the angle around the $z$-axis $\varphi$

\documentclass[border=3pt,tikz]{standalone}
\usepackage{physics}
\usepackage{tikz}
\usepackage{tikz-3dplot}
\usepackage{xcolor}
\usepackage[outline]{contour} % glow around text
\contourlength{0.9pt}
\usetikzlibrary{bending} % for arrow head angle

\colorlet{veccol}{green!50!black}
\colorlet{myred}{red!70!black}
\colorlet{myblue}{blue!70!black}
\colorlet{mydarkred}{red!40!black}
\colorlet{mydarkblue}{blue!30!black}
\colorlet{CMScol}{red!80!black}
\colorlet{ATLAScol}{blue!80!black}
\tikzset{>=latex} % for LaTeX arrow head
\tikzstyle{axis}=[->,thick,line cap=round]
\tikzstyle{detector}=[thick,draw=mydarkred,rotate around z=\ang]
\tikzstyle{beam pipe}=[draw=blue!20!black!50,fill=blue!20!black!10,rotate around z=\ang]
\tikzstyle{detector surface}=[red!60!black!60,opacity=0.06,rotate around z=\ang]
\usetikzlibrary{angles,quotes} % for pic (angle labels)
\newcommand*{\vv}[1]{\vec{\mkern0mu#1}} % aligned vector arrow


% CMS detector - right perspective
\tdplotsetmaincoords{70}{125} % to reset previous setting
\begin{tikzpicture}[scale=2.8,tdplot_main_coords,rotate around x=90]
  
  % VARIABLES
  \def\rvec{\L/2/cos(\thetavec)}
  \def\thetavec{17}
  \def\phivec{60}
  \def\L{3.1}     % detector length
  \def\R{0.75}    % detector cylinder radius
  \def\l{4.1}     % beam pipe length
  \def\r{0.04}    % beam pipe radius
  \def\rt{0.042}  % beam pipe radius + line thickness
  \def\xmax{1}    % maximum x axis
  \def\ymax{1.05} % maximum y axis
  \def\zmin{-\l/2-0.2} % minimum z axis
  \def\zmax{\l/2+0.3}  % maximum z axis
  \def\w{0.3}
  \coordinate (O) at (0,0,0);
  \coordinate (Z) at (0,0,\L/2);
  \tdplotsetcoord{O'}{0.8*\rvec}{\thetavec}{\phivec} % shifted origin
  \tdplotsetcoord{O''}{0.018}{90}{\phivec} % shifted origin
  \tdplotsetcoord{P}{\rvec}{\thetavec}{\phivec}
  
  % CYLINDER behind
  \def\ang{20} % rotate lines to simulate cylinder
  \fill[top color=red!50!black!4,bottom color=red!60!black!2,rotate around z=\ang]
    (0,\R,\L/2) --++ (0,0,-\L) arc(90:270:\R) --++ (0,0,\L) arc(270:90:\R) -- cycle;
  \fill[detector surface] % transverse plane at z=L/2
    (0,0,\L/2) --++ (0,\R,0) arc(90:270:\R) -- cycle;
  \fill[detector surface] % transverse plane at z=-L/2
    (0,0,-\L/2) --++ (0,\R,0) arc(90:270:\R) -- cycle;
  \tdplotdrawarc[detector,thin]{(0,0,\L/2)}{\R}{0}{360}{}{} % transverse plane at z=L/2
  \tdplotdrawarc[detector]{(0,0,-\L/2)}{\R}{0}{360}{}{} % transverse plane at z=-L/2
  \draw[detector,thin] % transverse plane at z=0
    (90-\ang:\R) arc (90-\ang:270:\R);
  \draw[detector] (0,0,-\L/2)++(90:\R) --++ (0,0,\L); % top horizontal
  \draw[detector] (0,0,-\L/2)++(-90:\R) --++ (0,0,\L); % bottom horizontal
  
  % BEAM PIPE
  \tdplotdrawarc[beam pipe]{(0,0,-\l/2)}{\r}{0}{360}{}{}
  \draw[beam pipe] % beam pipe, thinner in middle
    (0,\r,-\l/2) -- (0,\r,-0.2*\l) -- (90:0.5*\r)
    -- (0,\r,0.2*\l) -- (0,\r,0.5*\l) arc(90:-90:\r)
    -- (0,-\r,0.2*\l) -- (-90:0.5*\r) --
    (0,-\r,-0.2*\l) -- (0,-\r,-\l/2) arc(-90:90:\r);
  \draw[beam pipe] (0,0,\l/2) circle(\r);
  
  % AXES
  \draw[axis] (0,0,0) -- (0,0,\zmax) node[right=3,below=2]{$z$}; % long
  \draw[->,red] (O) -- (P) node[above left=-4] {$\vv{p}$};
  \fill[CMScol] (O) circle(0.5pt) node[right=1,below=1] {IP};
  \draw[axis,-] (0,0,\zmin) -- (0,0,-0.020); % long
  \draw[axis] (0,0.019,0) -- (0,\ymax,0) node[below left]{$y$};
  \draw[axis] (0.022,0,0) -- (\xmax,0,0) node[below=1,left=-2]{$x$};
  
  % LABELS
  \node[mydarkred,above] at (0,\ymax,0) {$\eta=0$};
  \node[mydarkred,above=6] at (0,\R,0.23*\L) {$\eta>0$};
  \node[mydarkred,above=6] at (0,\R,-0.3*\L) {$\eta<0$};
  \node[mydarkred,above=1,left] at (0,0,\zmax) {$\eta=\infty$};
  \node[mydarkred,below=1,right] at (0,0,\zmin) {$\eta=-\infty$};
  
  % VECTORS
  %\fill[radius=0.4,red] (P) circle;
  \draw[dashed,myred] (O'')  -- (Pxy);
  \draw[dashed,myred] (P)  -- (Pxy);
  \draw[dashed,myred] (Py) -- (Pxy);
  \draw[dashed,myred] (P) -- (Pz);
  \draw[->,red] (O') -- (P);
  
  % CYLINDER front
  \draw[beam pipe,fill=none] (0,\r,-\l/2) arc(90:-90:\r);
  \fill[detector surface] % transverse plane at z=L/2
    (0,\rt,\L/2) --++ (0,\R-\rt,0) arc(90:-90:\R) --++ (0,\R-\rt,0) arc(-90:90:\rt);
  \fill[detector surface] % transverse plane at z=-L/2
    (0,\rt,-\L/2) --++ (0,\R-\rt,0) arc(90:-90:\R) --++ (0,\R-\rt,0) arc(-90:90:\rt);
  \tdplotdrawarc[detector]{(0,0,\L/2)}{\R}{-90}{90}{}{}
  \tdplotdrawarc[detector]{(0,0,-\L/2)}{\R}{-90}{90}{}{}
  \draw[beam pipe,fill=none] (0,\r,\l/2) arc(90:-90:\r);
  \draw[detector,very thin] % transverse plane at z=0
    (90-\ang:\R) arc (90-\ang:-90:\R);
  
  % ANGLES
  \tdplotdrawarc[thick,red!57!black!3]{(O)}{0.2}{4}{0.7*\phivec}{}{} % contour
  \tdplotdrawarc[draw=none,opacity=0.8]{(O)}{0.2}{0}{\phivec} % transparant contour
    {below=4,left=-1,anchor=mid east}{\contour{red!55!black!3}{$\varphi$}}
  \tdplotdrawarc[->]{(O)}{0.2}{0}{\phivec}
    {below=4,left=-1,anchor=mid east}{$\varphi$}
  \tdplotdrawarc[->,rotate around z=\phivec-90,rotate around y=-90]
    {(O)}{1.05}{0}{\thetavec}{above=2,anchor=mid east}{$\theta$}
  \tdplotdrawarc[-{>[bend=1]},rotate around z=\phivec-90,rotate around y=-90,line cap=round]
    {(O)}{0.3}{90}{\thetavec}{above=4,right=2,anchor=mid east}{$\eta$}
  \draw[mydarkred,line cap=round] (0.004,0,\L/2) --++ (\R,0,0);
  \tdplotdrawarc[thick,red!60!black!4]{(Z)}{0.2}{4}{0.7*\phivec}{}{} % contour
  \tdplotdrawarc[draw=none]{(Z)}{0.2}{0}{\phivec} % transparant contour
    {below=4,left=-1,anchor=mid east,opacity=0.8}{\contour{red!60!black!4}{$\varphi$}}
  \tdplotdrawarc[->]{(Z)}{0.2}{0}{\phivec}
    {below=4,left=-1,anchor=mid east}{$\varphi$}
  
  % COMPASS - CMS-ATLAS axis has a ~12° declination (http://googlecompass.com)
%   \begin{scope}[shift={(1.3*\R,-\R,0.45*\L)},rotate around y=12]
%     \draw[<->,black!50] (-\w,0,0) -- (\w,0,0);
%     \draw[<->,black!50] (0,0,-\w) -- (0,0,\w);
%     \node[right=4,above,black!50,scale=0.6] at (-\w,0,0) {N};
%     \node[above=1,left=-1,green!20!black!50,scale=0.6] at (0,0,\w) {Jura};
%     \node[left=2,below=1,blue!30!black!50,scale=0.6] at (\w,0,0) {ATLAS};
%   \end{scope}
  \draw[->,thick,orange!30!black] (1.4*\w,-\R,0.1*\L) --++ (2*\w,0,0)
    node[below=1,left,scale=0.8,align=center] {center of\\[-1pt]the LHC};

\end{tikzpicture}

\subsection{Run 3 Specifics}
\begin{itemize}
    \item What was upgraded/added in Run 3?
    \item 
\end{itemize}

\subsection{Muon Forward Tracker}